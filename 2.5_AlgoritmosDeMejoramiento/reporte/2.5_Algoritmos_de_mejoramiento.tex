\documentclass[11pt]{article}

    \usepackage[breakable]{tcolorbox}
    \usepackage{parskip} % Stop auto-indenting (to mimic markdown behaviour)
    

    % Basic figure setup, for now with no caption control since it's done
    % automatically by Pandoc (which extracts ![](path) syntax from Markdown).
    \usepackage{graphicx}
    % Keep aspect ratio if custom image width or height is specified
    \setkeys{Gin}{keepaspectratio}
    % Maintain compatibility with old templates. Remove in nbconvert 6.0
    \let\Oldincludegraphics\includegraphics
    % Ensure that by default, figures have no caption (until we provide a
    % proper Figure object with a Caption API and a way to capture that
    % in the conversion process - todo).
    \usepackage{caption}
    \DeclareCaptionFormat{nocaption}{}
    \captionsetup{format=nocaption,aboveskip=0pt,belowskip=0pt}

    \usepackage{float}
    \floatplacement{figure}{H} % forces figures to be placed at the correct location
    \usepackage{xcolor} % Allow colors to be defined
    \usepackage{enumerate} % Needed for markdown enumerations to work
    \usepackage{geometry} % Used to adjust the document margins
    \usepackage{amsmath} % Equations
    \usepackage{amssymb} % Equations
    \usepackage{textcomp} % defines textquotesingle
    % Hack from http://tex.stackexchange.com/a/47451/13684:
    \AtBeginDocument{%
        \def\PYZsq{\textquotesingle}% Upright quotes in Pygmentized code
    }
    \usepackage{upquote} % Upright quotes for verbatim code
    \usepackage{eurosym} % defines \euro

    \usepackage{iftex}
    \ifPDFTeX
        \usepackage[T1]{fontenc}
        \IfFileExists{alphabeta.sty}{
              \usepackage{alphabeta}
          }{
              \usepackage[mathletters]{ucs}
              \usepackage[utf8x]{inputenc}
          }
    \else
        \usepackage{fontspec}
        \usepackage{unicode-math}
    \fi

    \usepackage{fancyvrb} % verbatim replacement that allows latex
    \usepackage{grffile} % extends the file name processing of package graphics
                         % to support a larger range
    \makeatletter % fix for old versions of grffile with XeLaTeX
    \@ifpackagelater{grffile}{2019/11/01}
    {
      % Do nothing on new versions
    }
    {
      \def\Gread@@xetex#1{%
        \IfFileExists{"\Gin@base".bb}%
        {\Gread@eps{\Gin@base.bb}}%
        {\Gread@@xetex@aux#1}%
      }
    }
    \makeatother
    \usepackage[Export]{adjustbox} % Used to constrain images to a maximum size
    \adjustboxset{max size={0.9\linewidth}{0.9\paperheight}}

    % The hyperref package gives us a pdf with properly built
    % internal navigation ('pdf bookmarks' for the table of contents,
    % internal cross-reference links, web links for URLs, etc.)
    \usepackage{hyperref}
    % The default LaTeX title has an obnoxious amount of whitespace. By default,
    % titling removes some of it. It also provides customization options.
    \usepackage{titling}
    \usepackage{longtable} % longtable support required by pandoc >1.10
    \usepackage{booktabs}  % table support for pandoc > 1.12.2
    \usepackage{array}     % table support for pandoc >= 2.11.3
    \usepackage{calc}      % table minipage width calculation for pandoc >= 2.11.1
    \usepackage[inline]{enumitem} % IRkernel/repr support (it uses the enumerate* environment)
    \usepackage[normalem]{ulem} % ulem is needed to support strikethroughs (\sout)
                                % normalem makes italics be italics, not underlines
    \usepackage{soul}      % strikethrough (\st) support for pandoc >= 3.0.0
    \usepackage{mathrsfs}
    

    
    % Colors for the hyperref package
    \definecolor{urlcolor}{rgb}{0,.145,.698}
    \definecolor{linkcolor}{rgb}{.71,0.21,0.01}
    \definecolor{citecolor}{rgb}{.12,.54,.11}

    % ANSI colors
    \definecolor{ansi-black}{HTML}{3E424D}
    \definecolor{ansi-black-intense}{HTML}{282C36}
    \definecolor{ansi-red}{HTML}{E75C58}
    \definecolor{ansi-red-intense}{HTML}{B22B31}
    \definecolor{ansi-green}{HTML}{00A250}
    \definecolor{ansi-green-intense}{HTML}{007427}
    \definecolor{ansi-yellow}{HTML}{DDB62B}
    \definecolor{ansi-yellow-intense}{HTML}{B27D12}
    \definecolor{ansi-blue}{HTML}{208FFB}
    \definecolor{ansi-blue-intense}{HTML}{0065CA}
    \definecolor{ansi-magenta}{HTML}{D160C4}
    \definecolor{ansi-magenta-intense}{HTML}{A03196}
    \definecolor{ansi-cyan}{HTML}{60C6C8}
    \definecolor{ansi-cyan-intense}{HTML}{258F8F}
    \definecolor{ansi-white}{HTML}{C5C1B4}
    \definecolor{ansi-white-intense}{HTML}{A1A6B2}
    \definecolor{ansi-default-inverse-fg}{HTML}{FFFFFF}
    \definecolor{ansi-default-inverse-bg}{HTML}{000000}

    % common color for the border for error outputs.
    \definecolor{outerrorbackground}{HTML}{FFDFDF}

    % commands and environments needed by pandoc snippets
    % extracted from the output of `pandoc -s`
    \providecommand{\tightlist}{%
      \setlength{\itemsep}{0pt}\setlength{\parskip}{0pt}}
    \DefineVerbatimEnvironment{Highlighting}{Verbatim}{commandchars=\\\{\}}
    % Add ',fontsize=\small' for more characters per line
    \newenvironment{Shaded}{}{}
    \newcommand{\KeywordTok}[1]{\textcolor[rgb]{0.00,0.44,0.13}{\textbf{{#1}}}}
    \newcommand{\DataTypeTok}[1]{\textcolor[rgb]{0.56,0.13,0.00}{{#1}}}
    \newcommand{\DecValTok}[1]{\textcolor[rgb]{0.25,0.63,0.44}{{#1}}}
    \newcommand{\BaseNTok}[1]{\textcolor[rgb]{0.25,0.63,0.44}{{#1}}}
    \newcommand{\FloatTok}[1]{\textcolor[rgb]{0.25,0.63,0.44}{{#1}}}
    \newcommand{\CharTok}[1]{\textcolor[rgb]{0.25,0.44,0.63}{{#1}}}
    \newcommand{\StringTok}[1]{\textcolor[rgb]{0.25,0.44,0.63}{{#1}}}
    \newcommand{\CommentTok}[1]{\textcolor[rgb]{0.38,0.63,0.69}{\textit{{#1}}}}
    \newcommand{\OtherTok}[1]{\textcolor[rgb]{0.00,0.44,0.13}{{#1}}}
    \newcommand{\AlertTok}[1]{\textcolor[rgb]{1.00,0.00,0.00}{\textbf{{#1}}}}
    \newcommand{\FunctionTok}[1]{\textcolor[rgb]{0.02,0.16,0.49}{{#1}}}
    \newcommand{\RegionMarkerTok}[1]{{#1}}
    \newcommand{\ErrorTok}[1]{\textcolor[rgb]{1.00,0.00,0.00}{\textbf{{#1}}}}
    \newcommand{\NormalTok}[1]{{#1}}

    % Additional commands for more recent versions of Pandoc
    \newcommand{\ConstantTok}[1]{\textcolor[rgb]{0.53,0.00,0.00}{{#1}}}
    \newcommand{\SpecialCharTok}[1]{\textcolor[rgb]{0.25,0.44,0.63}{{#1}}}
    \newcommand{\VerbatimStringTok}[1]{\textcolor[rgb]{0.25,0.44,0.63}{{#1}}}
    \newcommand{\SpecialStringTok}[1]{\textcolor[rgb]{0.73,0.40,0.53}{{#1}}}
    \newcommand{\ImportTok}[1]{{#1}}
    \newcommand{\DocumentationTok}[1]{\textcolor[rgb]{0.73,0.13,0.13}{\textit{{#1}}}}
    \newcommand{\AnnotationTok}[1]{\textcolor[rgb]{0.38,0.63,0.69}{\textbf{\textit{{#1}}}}}
    \newcommand{\CommentVarTok}[1]{\textcolor[rgb]{0.38,0.63,0.69}{\textbf{\textit{{#1}}}}}
    \newcommand{\VariableTok}[1]{\textcolor[rgb]{0.10,0.09,0.49}{{#1}}}
    \newcommand{\ControlFlowTok}[1]{\textcolor[rgb]{0.00,0.44,0.13}{\textbf{{#1}}}}
    \newcommand{\OperatorTok}[1]{\textcolor[rgb]{0.40,0.40,0.40}{{#1}}}
    \newcommand{\BuiltInTok}[1]{{#1}}
    \newcommand{\ExtensionTok}[1]{{#1}}
    \newcommand{\PreprocessorTok}[1]{\textcolor[rgb]{0.74,0.48,0.00}{{#1}}}
    \newcommand{\AttributeTok}[1]{\textcolor[rgb]{0.49,0.56,0.16}{{#1}}}
    \newcommand{\InformationTok}[1]{\textcolor[rgb]{0.38,0.63,0.69}{\textbf{\textit{{#1}}}}}
    \newcommand{\WarningTok}[1]{\textcolor[rgb]{0.38,0.63,0.69}{\textbf{\textit{{#1}}}}}
    \makeatletter
    \newsavebox\pandoc@box
    \newcommand*\pandocbounded[1]{%
      \sbox\pandoc@box{#1}%
      % scaling factors for width and height
      \Gscale@div\@tempa\textheight{\dimexpr\ht\pandoc@box+\dp\pandoc@box\relax}%
      \Gscale@div\@tempb\linewidth{\wd\pandoc@box}%
      % select the smaller of both
      \ifdim\@tempb\p@<\@tempa\p@
        \let\@tempa\@tempb
      \fi
      % scaling accordingly (\@tempa < 1)
      \ifdim\@tempa\p@<\p@
        \scalebox{\@tempa}{\usebox\pandoc@box}%
      % scaling not needed, use as it is
      \else
        \usebox{\pandoc@box}%
      \fi
    }
    \makeatother

    % Define a nice break command that doesn't care if a line doesn't already
    % exist.
    \def\br{\hspace*{\fill} \\* }
    % Math Jax compatibility definitions
    \def\gt{>}
    \def\lt{<}
    \let\Oldtex\TeX
    \let\Oldlatex\LaTeX
    \renewcommand{\TeX}{\textrm{\Oldtex}}
    \renewcommand{\LaTeX}{\textrm{\Oldlatex}}
    % Document parameters
    % Document title
    \title{2.5\_Algoritmos\_de\_mejoramiento}
    
    
    
    
    
    
    
% Pygments definitions
\makeatletter
\def\PY@reset{\let\PY@it=\relax \let\PY@bf=\relax%
    \let\PY@ul=\relax \let\PY@tc=\relax%
    \let\PY@bc=\relax \let\PY@ff=\relax}
\def\PY@tok#1{\csname PY@tok@#1\endcsname}
\def\PY@toks#1+{\ifx\relax#1\empty\else%
    \PY@tok{#1}\expandafter\PY@toks\fi}
\def\PY@do#1{\PY@bc{\PY@tc{\PY@ul{%
    \PY@it{\PY@bf{\PY@ff{#1}}}}}}}
\def\PY#1#2{\PY@reset\PY@toks#1+\relax+\PY@do{#2}}

\@namedef{PY@tok@w}{\def\PY@tc##1{\textcolor[rgb]{0.73,0.73,0.73}{##1}}}
\@namedef{PY@tok@c}{\let\PY@it=\textit\def\PY@tc##1{\textcolor[rgb]{0.24,0.48,0.48}{##1}}}
\@namedef{PY@tok@cp}{\def\PY@tc##1{\textcolor[rgb]{0.61,0.40,0.00}{##1}}}
\@namedef{PY@tok@k}{\let\PY@bf=\textbf\def\PY@tc##1{\textcolor[rgb]{0.00,0.50,0.00}{##1}}}
\@namedef{PY@tok@kp}{\def\PY@tc##1{\textcolor[rgb]{0.00,0.50,0.00}{##1}}}
\@namedef{PY@tok@kt}{\def\PY@tc##1{\textcolor[rgb]{0.69,0.00,0.25}{##1}}}
\@namedef{PY@tok@o}{\def\PY@tc##1{\textcolor[rgb]{0.40,0.40,0.40}{##1}}}
\@namedef{PY@tok@ow}{\let\PY@bf=\textbf\def\PY@tc##1{\textcolor[rgb]{0.67,0.13,1.00}{##1}}}
\@namedef{PY@tok@nb}{\def\PY@tc##1{\textcolor[rgb]{0.00,0.50,0.00}{##1}}}
\@namedef{PY@tok@nf}{\def\PY@tc##1{\textcolor[rgb]{0.00,0.00,1.00}{##1}}}
\@namedef{PY@tok@nc}{\let\PY@bf=\textbf\def\PY@tc##1{\textcolor[rgb]{0.00,0.00,1.00}{##1}}}
\@namedef{PY@tok@nn}{\let\PY@bf=\textbf\def\PY@tc##1{\textcolor[rgb]{0.00,0.00,1.00}{##1}}}
\@namedef{PY@tok@ne}{\let\PY@bf=\textbf\def\PY@tc##1{\textcolor[rgb]{0.80,0.25,0.22}{##1}}}
\@namedef{PY@tok@nv}{\def\PY@tc##1{\textcolor[rgb]{0.10,0.09,0.49}{##1}}}
\@namedef{PY@tok@no}{\def\PY@tc##1{\textcolor[rgb]{0.53,0.00,0.00}{##1}}}
\@namedef{PY@tok@nl}{\def\PY@tc##1{\textcolor[rgb]{0.46,0.46,0.00}{##1}}}
\@namedef{PY@tok@ni}{\let\PY@bf=\textbf\def\PY@tc##1{\textcolor[rgb]{0.44,0.44,0.44}{##1}}}
\@namedef{PY@tok@na}{\def\PY@tc##1{\textcolor[rgb]{0.41,0.47,0.13}{##1}}}
\@namedef{PY@tok@nt}{\let\PY@bf=\textbf\def\PY@tc##1{\textcolor[rgb]{0.00,0.50,0.00}{##1}}}
\@namedef{PY@tok@nd}{\def\PY@tc##1{\textcolor[rgb]{0.67,0.13,1.00}{##1}}}
\@namedef{PY@tok@s}{\def\PY@tc##1{\textcolor[rgb]{0.73,0.13,0.13}{##1}}}
\@namedef{PY@tok@sd}{\let\PY@it=\textit\def\PY@tc##1{\textcolor[rgb]{0.73,0.13,0.13}{##1}}}
\@namedef{PY@tok@si}{\let\PY@bf=\textbf\def\PY@tc##1{\textcolor[rgb]{0.64,0.35,0.47}{##1}}}
\@namedef{PY@tok@se}{\let\PY@bf=\textbf\def\PY@tc##1{\textcolor[rgb]{0.67,0.36,0.12}{##1}}}
\@namedef{PY@tok@sr}{\def\PY@tc##1{\textcolor[rgb]{0.64,0.35,0.47}{##1}}}
\@namedef{PY@tok@ss}{\def\PY@tc##1{\textcolor[rgb]{0.10,0.09,0.49}{##1}}}
\@namedef{PY@tok@sx}{\def\PY@tc##1{\textcolor[rgb]{0.00,0.50,0.00}{##1}}}
\@namedef{PY@tok@m}{\def\PY@tc##1{\textcolor[rgb]{0.40,0.40,0.40}{##1}}}
\@namedef{PY@tok@gh}{\let\PY@bf=\textbf\def\PY@tc##1{\textcolor[rgb]{0.00,0.00,0.50}{##1}}}
\@namedef{PY@tok@gu}{\let\PY@bf=\textbf\def\PY@tc##1{\textcolor[rgb]{0.50,0.00,0.50}{##1}}}
\@namedef{PY@tok@gd}{\def\PY@tc##1{\textcolor[rgb]{0.63,0.00,0.00}{##1}}}
\@namedef{PY@tok@gi}{\def\PY@tc##1{\textcolor[rgb]{0.00,0.52,0.00}{##1}}}
\@namedef{PY@tok@gr}{\def\PY@tc##1{\textcolor[rgb]{0.89,0.00,0.00}{##1}}}
\@namedef{PY@tok@ge}{\let\PY@it=\textit}
\@namedef{PY@tok@gs}{\let\PY@bf=\textbf}
\@namedef{PY@tok@ges}{\let\PY@bf=\textbf\let\PY@it=\textit}
\@namedef{PY@tok@gp}{\let\PY@bf=\textbf\def\PY@tc##1{\textcolor[rgb]{0.00,0.00,0.50}{##1}}}
\@namedef{PY@tok@go}{\def\PY@tc##1{\textcolor[rgb]{0.44,0.44,0.44}{##1}}}
\@namedef{PY@tok@gt}{\def\PY@tc##1{\textcolor[rgb]{0.00,0.27,0.87}{##1}}}
\@namedef{PY@tok@err}{\def\PY@bc##1{{\setlength{\fboxsep}{\string -\fboxrule}\fcolorbox[rgb]{1.00,0.00,0.00}{1,1,1}{\strut ##1}}}}
\@namedef{PY@tok@kc}{\let\PY@bf=\textbf\def\PY@tc##1{\textcolor[rgb]{0.00,0.50,0.00}{##1}}}
\@namedef{PY@tok@kd}{\let\PY@bf=\textbf\def\PY@tc##1{\textcolor[rgb]{0.00,0.50,0.00}{##1}}}
\@namedef{PY@tok@kn}{\let\PY@bf=\textbf\def\PY@tc##1{\textcolor[rgb]{0.00,0.50,0.00}{##1}}}
\@namedef{PY@tok@kr}{\let\PY@bf=\textbf\def\PY@tc##1{\textcolor[rgb]{0.00,0.50,0.00}{##1}}}
\@namedef{PY@tok@bp}{\def\PY@tc##1{\textcolor[rgb]{0.00,0.50,0.00}{##1}}}
\@namedef{PY@tok@fm}{\def\PY@tc##1{\textcolor[rgb]{0.00,0.00,1.00}{##1}}}
\@namedef{PY@tok@vc}{\def\PY@tc##1{\textcolor[rgb]{0.10,0.09,0.49}{##1}}}
\@namedef{PY@tok@vg}{\def\PY@tc##1{\textcolor[rgb]{0.10,0.09,0.49}{##1}}}
\@namedef{PY@tok@vi}{\def\PY@tc##1{\textcolor[rgb]{0.10,0.09,0.49}{##1}}}
\@namedef{PY@tok@vm}{\def\PY@tc##1{\textcolor[rgb]{0.10,0.09,0.49}{##1}}}
\@namedef{PY@tok@sa}{\def\PY@tc##1{\textcolor[rgb]{0.73,0.13,0.13}{##1}}}
\@namedef{PY@tok@sb}{\def\PY@tc##1{\textcolor[rgb]{0.73,0.13,0.13}{##1}}}
\@namedef{PY@tok@sc}{\def\PY@tc##1{\textcolor[rgb]{0.73,0.13,0.13}{##1}}}
\@namedef{PY@tok@dl}{\def\PY@tc##1{\textcolor[rgb]{0.73,0.13,0.13}{##1}}}
\@namedef{PY@tok@s2}{\def\PY@tc##1{\textcolor[rgb]{0.73,0.13,0.13}{##1}}}
\@namedef{PY@tok@sh}{\def\PY@tc##1{\textcolor[rgb]{0.73,0.13,0.13}{##1}}}
\@namedef{PY@tok@s1}{\def\PY@tc##1{\textcolor[rgb]{0.73,0.13,0.13}{##1}}}
\@namedef{PY@tok@mb}{\def\PY@tc##1{\textcolor[rgb]{0.40,0.40,0.40}{##1}}}
\@namedef{PY@tok@mf}{\def\PY@tc##1{\textcolor[rgb]{0.40,0.40,0.40}{##1}}}
\@namedef{PY@tok@mh}{\def\PY@tc##1{\textcolor[rgb]{0.40,0.40,0.40}{##1}}}
\@namedef{PY@tok@mi}{\def\PY@tc##1{\textcolor[rgb]{0.40,0.40,0.40}{##1}}}
\@namedef{PY@tok@il}{\def\PY@tc##1{\textcolor[rgb]{0.40,0.40,0.40}{##1}}}
\@namedef{PY@tok@mo}{\def\PY@tc##1{\textcolor[rgb]{0.40,0.40,0.40}{##1}}}
\@namedef{PY@tok@ch}{\let\PY@it=\textit\def\PY@tc##1{\textcolor[rgb]{0.24,0.48,0.48}{##1}}}
\@namedef{PY@tok@cm}{\let\PY@it=\textit\def\PY@tc##1{\textcolor[rgb]{0.24,0.48,0.48}{##1}}}
\@namedef{PY@tok@cpf}{\let\PY@it=\textit\def\PY@tc##1{\textcolor[rgb]{0.24,0.48,0.48}{##1}}}
\@namedef{PY@tok@c1}{\let\PY@it=\textit\def\PY@tc##1{\textcolor[rgb]{0.24,0.48,0.48}{##1}}}
\@namedef{PY@tok@cs}{\let\PY@it=\textit\def\PY@tc##1{\textcolor[rgb]{0.24,0.48,0.48}{##1}}}

\def\PYZbs{\char`\\}
\def\PYZus{\char`\_}
\def\PYZob{\char`\{}
\def\PYZcb{\char`\}}
\def\PYZca{\char`\^}
\def\PYZam{\char`\&}
\def\PYZlt{\char`\<}
\def\PYZgt{\char`\>}
\def\PYZsh{\char`\#}
\def\PYZpc{\char`\%}
\def\PYZdl{\char`\$}
\def\PYZhy{\char`\-}
\def\PYZsq{\char`\'}
\def\PYZdq{\char`\"}
\def\PYZti{\char`\~}
% for compatibility with earlier versions
\def\PYZat{@}
\def\PYZlb{[}
\def\PYZrb{]}
\makeatother


    % For linebreaks inside Verbatim environment from package fancyvrb.
    \makeatletter
        \newbox\Wrappedcontinuationbox
        \newbox\Wrappedvisiblespacebox
        \newcommand*\Wrappedvisiblespace {\textcolor{red}{\textvisiblespace}}
        \newcommand*\Wrappedcontinuationsymbol {\textcolor{red}{\llap{\tiny$\m@th\hookrightarrow$}}}
        \newcommand*\Wrappedcontinuationindent {3ex }
        \newcommand*\Wrappedafterbreak {\kern\Wrappedcontinuationindent\copy\Wrappedcontinuationbox}
        % Take advantage of the already applied Pygments mark-up to insert
        % potential linebreaks for TeX processing.
        %        {, <, #, %, $, ' and ": go to next line.
        %        _, }, ^, &, >, - and ~: stay at end of broken line.
        % Use of \textquotesingle for straight quote.
        \newcommand*\Wrappedbreaksatspecials {%
            \def\PYGZus{\discretionary{\char`\_}{\Wrappedafterbreak}{\char`\_}}%
            \def\PYGZob{\discretionary{}{\Wrappedafterbreak\char`\{}{\char`\{}}%
            \def\PYGZcb{\discretionary{\char`\}}{\Wrappedafterbreak}{\char`\}}}%
            \def\PYGZca{\discretionary{\char`\^}{\Wrappedafterbreak}{\char`\^}}%
            \def\PYGZam{\discretionary{\char`\&}{\Wrappedafterbreak}{\char`\&}}%
            \def\PYGZlt{\discretionary{}{\Wrappedafterbreak\char`\<}{\char`\<}}%
            \def\PYGZgt{\discretionary{\char`\>}{\Wrappedafterbreak}{\char`\>}}%
            \def\PYGZsh{\discretionary{}{\Wrappedafterbreak\char`\#}{\char`\#}}%
            \def\PYGZpc{\discretionary{}{\Wrappedafterbreak\char`\%}{\char`\%}}%
            \def\PYGZdl{\discretionary{}{\Wrappedafterbreak\char`\$}{\char`\$}}%
            \def\PYGZhy{\discretionary{\char`\-}{\Wrappedafterbreak}{\char`\-}}%
            \def\PYGZsq{\discretionary{}{\Wrappedafterbreak\textquotesingle}{\textquotesingle}}%
            \def\PYGZdq{\discretionary{}{\Wrappedafterbreak\char`\"}{\char`\"}}%
            \def\PYGZti{\discretionary{\char`\~}{\Wrappedafterbreak}{\char`\~}}%
        }
        % Some characters . , ; ? ! / are not pygmentized.
        % This macro makes them "active" and they will insert potential linebreaks
        \newcommand*\Wrappedbreaksatpunct {%
            \lccode`\~`\.\lowercase{\def~}{\discretionary{\hbox{\char`\.}}{\Wrappedafterbreak}{\hbox{\char`\.}}}%
            \lccode`\~`\,\lowercase{\def~}{\discretionary{\hbox{\char`\,}}{\Wrappedafterbreak}{\hbox{\char`\,}}}%
            \lccode`\~`\;\lowercase{\def~}{\discretionary{\hbox{\char`\;}}{\Wrappedafterbreak}{\hbox{\char`\;}}}%
            \lccode`\~`\:\lowercase{\def~}{\discretionary{\hbox{\char`\:}}{\Wrappedafterbreak}{\hbox{\char`\:}}}%
            \lccode`\~`\?\lowercase{\def~}{\discretionary{\hbox{\char`\?}}{\Wrappedafterbreak}{\hbox{\char`\?}}}%
            \lccode`\~`\!\lowercase{\def~}{\discretionary{\hbox{\char`\!}}{\Wrappedafterbreak}{\hbox{\char`\!}}}%
            \lccode`\~`\/\lowercase{\def~}{\discretionary{\hbox{\char`\/}}{\Wrappedafterbreak}{\hbox{\char`\/}}}%
            \catcode`\.\active
            \catcode`\,\active
            \catcode`\;\active
            \catcode`\:\active
            \catcode`\?\active
            \catcode`\!\active
            \catcode`\/\active
            \lccode`\~`\~
        }
    \makeatother

    \let\OriginalVerbatim=\Verbatim
    \makeatletter
    \renewcommand{\Verbatim}[1][1]{%
        %\parskip\z@skip
        \sbox\Wrappedcontinuationbox {\Wrappedcontinuationsymbol}%
        \sbox\Wrappedvisiblespacebox {\FV@SetupFont\Wrappedvisiblespace}%
        \def\FancyVerbFormatLine ##1{\hsize\linewidth
            \vtop{\raggedright\hyphenpenalty\z@\exhyphenpenalty\z@
                \doublehyphendemerits\z@\finalhyphendemerits\z@
                \strut ##1\strut}%
        }%
        % If the linebreak is at a space, the latter will be displayed as visible
        % space at end of first line, and a continuation symbol starts next line.
        % Stretch/shrink are however usually zero for typewriter font.
        \def\FV@Space {%
            \nobreak\hskip\z@ plus\fontdimen3\font minus\fontdimen4\font
            \discretionary{\copy\Wrappedvisiblespacebox}{\Wrappedafterbreak}
            {\kern\fontdimen2\font}%
        }%

        % Allow breaks at special characters using \PYG... macros.
        \Wrappedbreaksatspecials
        % Breaks at punctuation characters . , ; ? ! and / need catcode=\active
        \OriginalVerbatim[#1,codes*=\Wrappedbreaksatpunct]%
    }
    \makeatother

    % Exact colors from NB
    \definecolor{incolor}{HTML}{303F9F}
    \definecolor{outcolor}{HTML}{D84315}
    \definecolor{cellborder}{HTML}{CFCFCF}
    \definecolor{cellbackground}{HTML}{F7F7F7}

    % prompt
    \makeatletter
    \newcommand{\boxspacing}{\kern\kvtcb@left@rule\kern\kvtcb@boxsep}
    \makeatother
    \newcommand{\prompt}[4]{
        {\ttfamily\llap{{\color{#2}[#3]:\hspace{3pt}#4}}\vspace{-\baselineskip}}
    }
    

    
    % Prevent overflowing lines due to hard-to-break entities
    \sloppy
    % Setup hyperref package
    \hypersetup{
      breaklinks=true,  % so long urls are correctly broken across lines
      colorlinks=true,
      urlcolor=urlcolor,
      linkcolor=linkcolor,
      citecolor=citecolor,
      }
    % Slightly bigger margins than the latex defaults
    
    \geometry{verbose,tmargin=1in,bmargin=1in,lmargin=1in,rmargin=1in}
    
    

\begin{document}
    
    \maketitle
    
    

    
    \hypertarget{google-colab---algoritmos-de-mejoramiento-de-imuxe1genes-basado-por-pixeles}{%
\section{2.5 Google Colab - Algoritmos de mejoramiento de imágenes
basado por
pixeles}\label{google-colab---algoritmos-de-mejoramiento-de-imuxe1genes-basado-por-pixeles}}

\hypertarget{computer-vision-for-images-and-video-tc4033---group-10}{%
\subsubsection{Computer Vision for Images and Video (TC4033 - Group
10)}\label{computer-vision-for-images-and-video-tc4033---group-10}}

\begin{itemize}
\tightlist
\item
  \textbf{School}: School of Engineering and Sciences, Tecnológico de
  Monterrey
\item
  \textbf{Lead Professor}: Dr.~Gilberto Ochoa Ruiz
\item
  \textbf{Assistant Professor}: MIP Ma. del Refugio Melendez Alfaro
\item
  \textbf{Tutor Professor}: Iván Reyes Amezcua
\item
  \textbf{Team 5}:

  \begin{itemize}
  \tightlist
  \item
    Benjamin Cisneros Barraza - A01796323
  \item
    Jonatan Israel Meza Mendoza - A01275322
  \item
    Carlos Pano Hernández - A01066264
  \item
    Edgar Omar Cruz Mendoza - A01795590
  \end{itemize}
\item
  \textbf{Date}: Sunday, September 21th, 2025
\end{itemize}

    \hypertarget{importing-libraries}{%
\subsection{\texorpdfstring{Importing Libraries
}{Importing Libraries }}\label{importing-libraries}}

    \begin{tcolorbox}[breakable, size=fbox, boxrule=1pt, pad at break*=1mm,colback=cellbackground, colframe=cellborder]
\prompt{In}{incolor}{18}{\boxspacing}
\begin{Verbatim}[commandchars=\\\{\}]
\PY{k+kn}{from}\PY{+w}{ }\PY{n+nn}{matplotlib}\PY{+w}{ }\PY{k+kn}{import} \PY{n}{image} \PY{k}{as} \PY{n}{mpimg}
\PY{k+kn}{import}\PY{+w}{ }\PY{n+nn}{matplotlib}\PY{n+nn}{.}\PY{n+nn}{pyplot}\PY{+w}{ }\PY{k}{as}\PY{+w}{ }\PY{n+nn}{plt}
\PY{k+kn}{import}\PY{+w}{ }\PY{n+nn}{numpy}\PY{+w}{ }\PY{k}{as}\PY{+w}{ }\PY{n+nn}{np}
\PY{k+kn}{import}\PY{+w}{ }\PY{n+nn}{cv2}
\end{Verbatim}
\end{tcolorbox}

    \hypertarget{loading-images}{%
\subsection{\texorpdfstring{Loading Images
}{Loading Images }}\label{loading-images}}

    \begin{tcolorbox}[breakable, size=fbox, boxrule=1pt, pad at break*=1mm,colback=cellbackground, colframe=cellborder]
\prompt{In}{incolor}{19}{\boxspacing}
\begin{Verbatim}[commandchars=\\\{\}]
\PY{c+c1}{\PYZsh{} Load images}
\PY{k}{try}\PY{p}{:}
    \PY{n}{img1} \PY{o}{=} \PY{n}{mpimg}\PY{o}{.}\PY{n}{imread}\PY{p}{(}\PY{l+s+s1}{\PYZsq{}}\PY{l+s+s1}{data/img1.jpg}\PY{l+s+s1}{\PYZsq{}}\PY{p}{)}
    \PY{n}{img2} \PY{o}{=} \PY{n}{mpimg}\PY{o}{.}\PY{n}{imread}\PY{p}{(}\PY{l+s+s1}{\PYZsq{}}\PY{l+s+s1}{data/img2.jpg}\PY{l+s+s1}{\PYZsq{}}\PY{p}{)}
    \PY{n}{img3} \PY{o}{=} \PY{n}{mpimg}\PY{o}{.}\PY{n}{imread}\PY{p}{(}\PY{l+s+s1}{\PYZsq{}}\PY{l+s+s1}{data/img3.jpg}\PY{l+s+s1}{\PYZsq{}}\PY{p}{)}
    \PY{n}{img10} \PY{o}{=} \PY{n}{mpimg}\PY{o}{.}\PY{n}{imread}\PY{p}{(}\PY{l+s+s1}{\PYZsq{}}\PY{l+s+s1}{data/img10.jpg}\PY{l+s+s1}{\PYZsq{}}\PY{p}{)}
    \PY{n+nb}{print}\PY{p}{(}\PY{l+s+s1}{\PYZsq{}}\PY{l+s+s1}{Images loaded successfully!}\PY{l+s+s1}{\PYZsq{}}\PY{p}{)}
    \PY{n+nb}{print}\PY{p}{(}\PY{l+s+s1}{\PYZsq{}}\PY{l+s+s1}{Image 1 shape:}\PY{l+s+s1}{\PYZsq{}}\PY{p}{,} \PY{n}{img1}\PY{o}{.}\PY{n}{shape}\PY{p}{)}
    \PY{n+nb}{print}\PY{p}{(}\PY{l+s+s1}{\PYZsq{}}\PY{l+s+s1}{Image 2 shape:}\PY{l+s+s1}{\PYZsq{}}\PY{p}{,} \PY{n}{img2}\PY{o}{.}\PY{n}{shape}\PY{p}{)}
    \PY{n+nb}{print}\PY{p}{(}\PY{l+s+s1}{\PYZsq{}}\PY{l+s+s1}{Image 3 shape:}\PY{l+s+s1}{\PYZsq{}}\PY{p}{,} \PY{n}{img3}\PY{o}{.}\PY{n}{shape}\PY{p}{)}
    \PY{n+nb}{print}\PY{p}{(}\PY{l+s+s1}{\PYZsq{}}\PY{l+s+s1}{Image 10 shape:}\PY{l+s+s1}{\PYZsq{}}\PY{p}{,} \PY{n}{img10}\PY{o}{.}\PY{n}{shape}\PY{p}{)}
\PY{k}{except} \PY{n+ne}{FileNotFoundError} \PY{k}{as} \PY{n}{e}\PY{p}{:}
    \PY{n+nb}{print}\PY{p}{(}\PY{l+s+sa}{f}\PY{l+s+s2}{\PYZdq{}}\PY{l+s+s2}{Error loading images: }\PY{l+s+si}{\PYZob{}}\PY{n}{e}\PY{l+s+si}{\PYZcb{}}\PY{l+s+s2}{. Please ensure the }\PY{l+s+s2}{\PYZsq{}}\PY{l+s+s2}{data}\PY{l+s+s2}{\PYZsq{}}\PY{l+s+s2}{ folder is in the same directory as this notebook and contains the required images.}\PY{l+s+s2}{\PYZdq{}}\PY{p}{)}
    \PY{c+c1}{\PYZsh{} Handle gracefully if images are not found}
    \PY{n}{img1}\PY{p}{,} \PY{n}{img2}\PY{p}{,} \PY{n}{img3}\PY{p}{,} \PY{n}{img10} \PY{o}{=} \PY{k+kc}{None}\PY{p}{,} \PY{k+kc}{None}\PY{p}{,} \PY{k+kc}{None}\PY{p}{,} \PY{k+kc}{None}
\end{Verbatim}
\end{tcolorbox}

    \begin{Verbatim}[commandchars=\\\{\}]
Images loaded successfully!
Image 1 shape: (3464, 4618, 3)
Image 2 shape: (3024, 4032, 3)
Image 3 shape: (4032, 3024, 3)
Image 10 shape: (256, 256, 3)
    \end{Verbatim}

    \begin{tcolorbox}[breakable, size=fbox, boxrule=1pt, pad at break*=1mm,colback=cellbackground, colframe=cellborder]
\prompt{In}{incolor}{20}{\boxspacing}
\begin{Verbatim}[commandchars=\\\{\}]
\PY{c+c1}{\PYZsh{} Display the selected images}
\PY{k}{if} \PY{n}{img1} \PY{o+ow}{is} \PY{o+ow}{not} \PY{k+kc}{None} \PY{o+ow}{and} \PY{n}{img2} \PY{o+ow}{is} \PY{o+ow}{not} \PY{k+kc}{None} \PY{o+ow}{and} \PY{n}{img3} \PY{o+ow}{is} \PY{o+ow}{not} \PY{k+kc}{None}\PY{p}{:}
    \PY{n}{fig}\PY{p}{,} \PY{n}{axes} \PY{o}{=} \PY{n}{plt}\PY{o}{.}\PY{n}{subplots}\PY{p}{(}\PY{l+m+mi}{1}\PY{p}{,} \PY{l+m+mi}{3}\PY{p}{,} \PY{n}{figsize}\PY{o}{=}\PY{p}{(}\PY{l+m+mi}{15}\PY{p}{,} \PY{l+m+mi}{5}\PY{p}{)}\PY{p}{)}

    \PY{n}{axes}\PY{p}{[}\PY{l+m+mi}{0}\PY{p}{]}\PY{o}{.}\PY{n}{imshow}\PY{p}{(}\PY{n}{img1}\PY{p}{)}
    \PY{n}{axes}\PY{p}{[}\PY{l+m+mi}{0}\PY{p}{]}\PY{o}{.}\PY{n}{set\PYZus{}title}\PY{p}{(}\PY{l+s+s1}{\PYZsq{}}\PY{l+s+s1}{Image 1}\PY{l+s+s1}{\PYZsq{}}\PY{p}{)}
    \PY{n}{axes}\PY{p}{[}\PY{l+m+mi}{0}\PY{p}{]}\PY{o}{.}\PY{n}{axis}\PY{p}{(}\PY{l+s+s1}{\PYZsq{}}\PY{l+s+s1}{off}\PY{l+s+s1}{\PYZsq{}}\PY{p}{)}

    \PY{n}{axes}\PY{p}{[}\PY{l+m+mi}{1}\PY{p}{]}\PY{o}{.}\PY{n}{imshow}\PY{p}{(}\PY{n}{img2}\PY{p}{)}
    \PY{n}{axes}\PY{p}{[}\PY{l+m+mi}{1}\PY{p}{]}\PY{o}{.}\PY{n}{set\PYZus{}title}\PY{p}{(}\PY{l+s+s1}{\PYZsq{}}\PY{l+s+s1}{Image 2}\PY{l+s+s1}{\PYZsq{}}\PY{p}{)}
    \PY{n}{axes}\PY{p}{[}\PY{l+m+mi}{1}\PY{p}{]}\PY{o}{.}\PY{n}{axis}\PY{p}{(}\PY{l+s+s1}{\PYZsq{}}\PY{l+s+s1}{off}\PY{l+s+s1}{\PYZsq{}}\PY{p}{)}

    \PY{n}{axes}\PY{p}{[}\PY{l+m+mi}{2}\PY{p}{]}\PY{o}{.}\PY{n}{imshow}\PY{p}{(}\PY{n}{img3}\PY{p}{)}
    \PY{n}{axes}\PY{p}{[}\PY{l+m+mi}{2}\PY{p}{]}\PY{o}{.}\PY{n}{set\PYZus{}title}\PY{p}{(}\PY{l+s+s1}{\PYZsq{}}\PY{l+s+s1}{Image 3}\PY{l+s+s1}{\PYZsq{}}\PY{p}{)}
    \PY{n}{axes}\PY{p}{[}\PY{l+m+mi}{2}\PY{p}{]}\PY{o}{.}\PY{n}{axis}\PY{p}{(}\PY{l+s+s1}{\PYZsq{}}\PY{l+s+s1}{off}\PY{l+s+s1}{\PYZsq{}}\PY{p}{)}

    \PY{n}{plt}\PY{o}{.}\PY{n}{tight\PYZus{}layout}\PY{p}{(}\PY{p}{)}
    \PY{n}{plt}\PY{o}{.}\PY{n}{show}\PY{p}{(}\PY{p}{)}
\end{Verbatim}
\end{tcolorbox}

    \begin{center}
    \adjustimage{max size={0.9\linewidth}{0.9\paperheight}}{output_5_0.png}
    \end{center}
    { \hspace*{\fill} \\}
    
    \hypertarget{exercise-1}{%
\subsubsection{Exercise 1}\label{exercise-1}}

Research and implement the tile-based histogram equalization method.

In these methods, the image is partitioned into different windows (i.e.,
tiles), and histograms are calculated independently for each window,
applying the correction to each tile. There is a trade-off between the
window size and computational complexity, making this method more
complex than the simple approach.

Implement this version and perform some tests by varying the window
size. Another drawback is that you may observe noticeable differences in
contrast between the blocks; how could this be improved?

\hypertarget{section}{%
\section{-------------------------------}\label{section}}

    \hypertarget{tile-based-histogram-equalization-clahe}{%
\subsection{Tile-Based Histogram Equalization
(CLAHE)}\label{tile-based-histogram-equalization-clahe}}

\textbf{Tile-based histogram equalization}, specifically the
\textbf{Contrast-Limited Adaptive Histogram Equalization (CLAHE)}
method, is a technique to improve contrast in images. Unlike global
histogram equalization, which calculates a single histogram for the
entire image, CLAHE divides the image into smaller, non-overlapping
regions called \textbf{tiles}.

The CLAHE method addresses the limitations of simple adaptive histogram
equalization (AHE) by preventing the amplification of noise and ensuring
smooth transitions between tiles. This method works by:

\begin{itemize}
\item
  \textbf{Tiling:} The input image is divided into a grid of rectangular
  tiles. The size of these tiles is a crucial parameter. A larger tile
  size results in a more uniform equalization, similar to the global
  method, while a smaller size provides more localized contrast
  enhancement.
\item
  \textbf{Local Equalization:} A histogram is computed for each tile. A
  key step in CLAHE is to \textbf{clip} this histogram at a predefined
  limit. This prevents any single intensity value from having a
  disproportionately high frequency, which would cause over-enhancement
  and noise amplification. After clipping, the histogram is normalized,
  and a separate mapping function is generated for each tile.
\item
  \textbf{Interpolation:} To eliminate the artificial boundaries that
  can appear between tiles due to independent processing, CLAHE uses
  \textbf{bilinear interpolation}. This process smooths the transition
  between adjacent tiles by blending the equalization results from the
  surrounding tiles, creating a seamless and natural-looking final
  image.
\end{itemize}

This approach is particularly effective for images where the contrast
varies significantly across different regions, revealing details in both
bright and dark areas simultaneously.

    \hypertarget{implementation-and-parameter-testing}{%
\subsubsection{Implementation and Parameter
Testing}\label{implementation-and-parameter-testing}}

    \begin{tcolorbox}[breakable, size=fbox, boxrule=1pt, pad at break*=1mm,colback=cellbackground, colframe=cellborder]
\prompt{In}{incolor}{21}{\boxspacing}
\begin{Verbatim}[commandchars=\\\{\}]
\PY{c+c1}{\PYZsh{} Creating our CLAHE function. Receives an image, a clip limit and a tile grid size.}
\PY{k}{def}\PY{+w}{ }\PY{n+nf}{apply\PYZus{}clahe}\PY{p}{(}\PY{n}{image}\PY{p}{,} \PY{n}{clip\PYZus{}limit}\PY{o}{=}\PY{l+m+mf}{2.0}\PY{p}{,} \PY{n}{tile\PYZus{}grid\PYZus{}size}\PY{o}{=}\PY{p}{(}\PY{l+m+mi}{8}\PY{p}{,} \PY{l+m+mi}{8}\PY{p}{)}\PY{p}{)}\PY{p}{:}
\PY{+w}{    }\PY{l+s+sd}{\PYZdq{}\PYZdq{}\PYZdq{}}
\PY{l+s+sd}{    Applies Contrast\PYZhy{}Limited Adaptive Histogram Equalization (CLAHE).}

\PY{l+s+sd}{    Parameters:}
\PY{l+s+sd}{    \PYZhy{} image: The input grayscale image.}
\PY{l+s+sd}{    \PYZhy{} clip\PYZus{}limit: The contrast clipping threshold. A higher value means more contrast.}
\PY{l+s+sd}{    \PYZhy{} tile\PYZus{}grid\PYZus{}size: The size of the tile grid (width, height).}

\PY{l+s+sd}{    Returns:}
\PY{l+s+sd}{    \PYZhy{} The processed image.}
\PY{l+s+sd}{    \PYZdq{}\PYZdq{}\PYZdq{}}
    \PY{c+c1}{\PYZsh{} Create a CLAHE object}
    \PY{n}{clahe} \PY{o}{=} \PY{n}{cv2}\PY{o}{.}\PY{n}{createCLAHE}\PY{p}{(}\PY{n}{clipLimit}\PY{o}{=}\PY{n}{clip\PYZus{}limit}\PY{p}{,} \PY{n}{tileGridSize}\PY{o}{=}\PY{n}{tile\PYZus{}grid\PYZus{}size}\PY{p}{)}

    \PY{c+c1}{\PYZsh{} Apply CLAHE to the image}
    \PY{n}{processed\PYZus{}image} \PY{o}{=} \PY{n}{clahe}\PY{o}{.}\PY{n}{apply}\PY{p}{(}\PY{n}{image}\PY{p}{)}

    \PY{k}{return} \PY{n}{processed\PYZus{}image}
\end{Verbatim}
\end{tcolorbox}

    \begin{tcolorbox}[breakable, size=fbox, boxrule=1pt, pad at break*=1mm,colback=cellbackground, colframe=cellborder]
\prompt{In}{incolor}{22}{\boxspacing}
\begin{Verbatim}[commandchars=\\\{\}]
\PY{k}{if} \PY{n}{img1} \PY{o+ow}{is} \PY{o+ow}{not} \PY{k+kc}{None} \PY{o+ow}{and} \PY{n}{img2} \PY{o+ow}{is} \PY{o+ow}{not} \PY{k+kc}{None} \PY{o+ow}{and} \PY{n}{img3} \PY{o+ow}{is} \PY{o+ow}{not} \PY{k+kc}{None}\PY{p}{:}
    \PY{k}{try}\PY{p}{:}
        \PY{c+c1}{\PYZsh{} Convert images to grayscale for CLAHE processing}
        \PY{n}{img1\PYZus{}gray} \PY{o}{=} \PY{n}{cv2}\PY{o}{.}\PY{n}{cvtColor}\PY{p}{(}\PY{n}{img1}\PY{p}{,} \PY{n}{cv2}\PY{o}{.}\PY{n}{COLOR\PYZus{}RGB2GRAY}\PY{p}{)}
        \PY{n}{img2\PYZus{}gray} \PY{o}{=} \PY{n}{cv2}\PY{o}{.}\PY{n}{cvtColor}\PY{p}{(}\PY{n}{img2}\PY{p}{,} \PY{n}{cv2}\PY{o}{.}\PY{n}{COLOR\PYZus{}RGB2GRAY}\PY{p}{)}
        \PY{n}{img3\PYZus{}gray} \PY{o}{=} \PY{n}{cv2}\PY{o}{.}\PY{n}{cvtColor}\PY{p}{(}\PY{n}{img3}\PY{p}{,} \PY{n}{cv2}\PY{o}{.}\PY{n}{COLOR\PYZus{}RGB2GRAY}\PY{p}{)}
        
        \PY{c+c1}{\PYZsh{} Vary the window size and contrast limits for testing}
        \PY{n}{sizes} \PY{o}{=} \PY{p}{[}\PY{p}{(}\PY{l+m+mi}{4}\PY{p}{,} \PY{l+m+mi}{4}\PY{p}{)}\PY{p}{,} \PY{p}{(}\PY{l+m+mi}{8}\PY{p}{,} \PY{l+m+mi}{8}\PY{p}{)}\PY{p}{,} \PY{p}{(}\PY{l+m+mi}{16}\PY{p}{,} \PY{l+m+mi}{16}\PY{p}{)}\PY{p}{]}
        \PY{n}{clip\PYZus{}limits} \PY{o}{=} \PY{p}{[}\PY{l+m+mf}{1.0}\PY{p}{,} \PY{l+m+mf}{2.0}\PY{p}{,} \PY{l+m+mf}{4.0}\PY{p}{]}
        
        \PY{n}{images} \PY{o}{=} \PY{p}{[}\PY{n}{img1\PYZus{}gray}\PY{p}{,} \PY{n}{img2\PYZus{}gray}\PY{p}{,} \PY{n}{img3\PYZus{}gray}\PY{p}{]}
        \PY{n}{image\PYZus{}names} \PY{o}{=} \PY{p}{[}\PY{l+s+s1}{\PYZsq{}}\PY{l+s+s1}{Image 1}\PY{l+s+s1}{\PYZsq{}}\PY{p}{,} \PY{l+s+s1}{\PYZsq{}}\PY{l+s+s1}{Image 2}\PY{l+s+s1}{\PYZsq{}}\PY{p}{,} \PY{l+s+s1}{\PYZsq{}}\PY{l+s+s1}{Image 3}\PY{l+s+s1}{\PYZsq{}}\PY{p}{]}
        
        \PY{k}{for} \PY{n}{img\PYZus{}idx}\PY{p}{,} \PY{p}{(}\PY{n}{img\PYZus{}gray}\PY{p}{,} \PY{n}{img\PYZus{}name}\PY{p}{)} \PY{o+ow}{in} \PY{n+nb}{enumerate}\PY{p}{(}\PY{n+nb}{zip}\PY{p}{(}\PY{n}{images}\PY{p}{,} \PY{n}{image\PYZus{}names}\PY{p}{)}\PY{p}{)}\PY{p}{:}
            \PY{n+nb}{print}\PY{p}{(}\PY{l+s+sa}{f}\PY{l+s+s2}{\PYZdq{}}\PY{l+s+se}{\PYZbs{}n}\PY{l+s+s2}{Processing }\PY{l+s+si}{\PYZob{}}\PY{n}{img\PYZus{}name}\PY{l+s+si}{\PYZcb{}}\PY{l+s+s2}{...}\PY{l+s+s2}{\PYZdq{}}\PY{p}{)}
            
            \PY{c+c1}{\PYZsh{} Create a comprehensive figure with images and histograms}
            \PY{n}{fig} \PY{o}{=} \PY{n}{plt}\PY{o}{.}\PY{n}{figure}\PY{p}{(}\PY{n}{figsize}\PY{o}{=}\PY{p}{(}\PY{l+m+mi}{20}\PY{p}{,} \PY{l+m+mi}{25}\PY{p}{)}\PY{p}{)}
            
            \PY{c+c1}{\PYZsh{} Create grid layout: 2 columns (image + histogram) for each row}
            \PY{n}{gs} \PY{o}{=} \PY{n}{fig}\PY{o}{.}\PY{n}{add\PYZus{}gridspec}\PY{p}{(}\PY{n+nb}{len}\PY{p}{(}\PY{n}{sizes}\PY{p}{)} \PY{o}{*} \PY{n+nb}{len}\PY{p}{(}\PY{n}{clip\PYZus{}limits}\PY{p}{)} \PY{o}{+} \PY{l+m+mi}{1}\PY{p}{,} \PY{l+m+mi}{2}\PY{p}{,} \PY{n}{hspace}\PY{o}{=}\PY{l+m+mf}{0.3}\PY{p}{,} \PY{n}{wspace}\PY{o}{=}\PY{l+m+mf}{0.3}\PY{p}{)}
            
            \PY{c+c1}{\PYZsh{} Original image and histogram}
            \PY{n}{ax\PYZus{}img} \PY{o}{=} \PY{n}{fig}\PY{o}{.}\PY{n}{add\PYZus{}subplot}\PY{p}{(}\PY{n}{gs}\PY{p}{[}\PY{l+m+mi}{0}\PY{p}{,} \PY{l+m+mi}{0}\PY{p}{]}\PY{p}{)}
            \PY{n}{ax\PYZus{}hist} \PY{o}{=} \PY{n}{fig}\PY{o}{.}\PY{n}{add\PYZus{}subplot}\PY{p}{(}\PY{n}{gs}\PY{p}{[}\PY{l+m+mi}{0}\PY{p}{,} \PY{l+m+mi}{1}\PY{p}{]}\PY{p}{)}
            
            \PY{n}{ax\PYZus{}img}\PY{o}{.}\PY{n}{imshow}\PY{p}{(}\PY{n}{img\PYZus{}gray}\PY{p}{,} \PY{n}{cmap}\PY{o}{=}\PY{l+s+s1}{\PYZsq{}}\PY{l+s+s1}{gray}\PY{l+s+s1}{\PYZsq{}}\PY{p}{)}
            \PY{n}{ax\PYZus{}img}\PY{o}{.}\PY{n}{set\PYZus{}title}\PY{p}{(}\PY{l+s+sa}{f}\PY{l+s+s2}{\PYZdq{}}\PY{l+s+si}{\PYZob{}}\PY{n}{img\PYZus{}name}\PY{l+s+si}{\PYZcb{}}\PY{l+s+s2}{ \PYZhy{} Original (Grayscale)}\PY{l+s+s2}{\PYZdq{}}\PY{p}{,} \PY{n}{fontsize}\PY{o}{=}\PY{l+m+mi}{14}\PY{p}{,} \PY{n}{fontweight}\PY{o}{=}\PY{l+s+s1}{\PYZsq{}}\PY{l+s+s1}{bold}\PY{l+s+s1}{\PYZsq{}}\PY{p}{)}
            \PY{n}{ax\PYZus{}img}\PY{o}{.}\PY{n}{axis}\PY{p}{(}\PY{l+s+s1}{\PYZsq{}}\PY{l+s+s1}{off}\PY{l+s+s1}{\PYZsq{}}\PY{p}{)}
            
            \PY{c+c1}{\PYZsh{} Original histogram}
            \PY{n}{ax\PYZus{}hist}\PY{o}{.}\PY{n}{hist}\PY{p}{(}\PY{n}{img\PYZus{}gray}\PY{o}{.}\PY{n}{ravel}\PY{p}{(}\PY{p}{)}\PY{p}{,} \PY{n}{bins}\PY{o}{=}\PY{l+m+mi}{256}\PY{p}{,} \PY{n+nb}{range}\PY{o}{=}\PY{p}{(}\PY{l+m+mi}{0}\PY{p}{,} \PY{l+m+mi}{255}\PY{p}{)}\PY{p}{,} \PY{n}{color}\PY{o}{=}\PY{l+s+s1}{\PYZsq{}}\PY{l+s+s1}{gray}\PY{l+s+s1}{\PYZsq{}}\PY{p}{,} \PY{n}{alpha}\PY{o}{=}\PY{l+m+mf}{0.7}\PY{p}{)}
            \PY{n}{ax\PYZus{}hist}\PY{o}{.}\PY{n}{set\PYZus{}title}\PY{p}{(}\PY{l+s+sa}{f}\PY{l+s+s2}{\PYZdq{}}\PY{l+s+s2}{Original Histogram}\PY{l+s+s2}{\PYZdq{}}\PY{p}{,} \PY{n}{fontsize}\PY{o}{=}\PY{l+m+mi}{12}\PY{p}{)}
            \PY{n}{ax\PYZus{}hist}\PY{o}{.}\PY{n}{set\PYZus{}xlabel}\PY{p}{(}\PY{l+s+s1}{\PYZsq{}}\PY{l+s+s1}{Pixel Intensity}\PY{l+s+s1}{\PYZsq{}}\PY{p}{)}
            \PY{n}{ax\PYZus{}hist}\PY{o}{.}\PY{n}{set\PYZus{}ylabel}\PY{p}{(}\PY{l+s+s1}{\PYZsq{}}\PY{l+s+s1}{Frequency}\PY{l+s+s1}{\PYZsq{}}\PY{p}{)}
            \PY{n}{ax\PYZus{}hist}\PY{o}{.}\PY{n}{grid}\PY{p}{(}\PY{k+kc}{True}\PY{p}{,} \PY{n}{alpha}\PY{o}{=}\PY{l+m+mf}{0.3}\PY{p}{)}
            
            \PY{n}{plot\PYZus{}index} \PY{o}{=} \PY{l+m+mi}{1}
            
            \PY{k}{for} \PY{n}{size} \PY{o+ow}{in} \PY{n}{sizes}\PY{p}{:}
                \PY{k}{for} \PY{n}{clip} \PY{o+ow}{in} \PY{n}{clip\PYZus{}limits}\PY{p}{:}
                    \PY{n}{processed\PYZus{}image} \PY{o}{=} \PY{n}{apply\PYZus{}clahe}\PY{p}{(}\PY{n}{img\PYZus{}gray}\PY{p}{,} \PY{n}{clip\PYZus{}limit}\PY{o}{=}\PY{n}{clip}\PY{p}{,} \PY{n}{tile\PYZus{}grid\PYZus{}size}\PY{o}{=}\PY{n}{size}\PY{p}{)}
                    
                    \PY{c+c1}{\PYZsh{} Image subplot}
                    \PY{n}{ax\PYZus{}img} \PY{o}{=} \PY{n}{fig}\PY{o}{.}\PY{n}{add\PYZus{}subplot}\PY{p}{(}\PY{n}{gs}\PY{p}{[}\PY{n}{plot\PYZus{}index}\PY{p}{,} \PY{l+m+mi}{0}\PY{p}{]}\PY{p}{)}
                    \PY{n}{ax\PYZus{}img}\PY{o}{.}\PY{n}{imshow}\PY{p}{(}\PY{n}{processed\PYZus{}image}\PY{p}{,} \PY{n}{cmap}\PY{o}{=}\PY{l+s+s1}{\PYZsq{}}\PY{l+s+s1}{gray}\PY{l+s+s1}{\PYZsq{}}\PY{p}{)}
                    \PY{n}{ax\PYZus{}img}\PY{o}{.}\PY{n}{set\PYZus{}title}\PY{p}{(}\PY{l+s+sa}{f}\PY{l+s+s2}{\PYZdq{}}\PY{l+s+s2}{CLAHE: Tile }\PY{l+s+si}{\PYZob{}}\PY{n}{size}\PY{l+s+si}{\PYZcb{}}\PY{l+s+s2}{, Clip }\PY{l+s+si}{\PYZob{}}\PY{n}{clip}\PY{l+s+si}{\PYZcb{}}\PY{l+s+s2}{\PYZdq{}}\PY{p}{,} \PY{n}{fontsize}\PY{o}{=}\PY{l+m+mi}{12}\PY{p}{)}
                    \PY{n}{ax\PYZus{}img}\PY{o}{.}\PY{n}{axis}\PY{p}{(}\PY{l+s+s1}{\PYZsq{}}\PY{l+s+s1}{off}\PY{l+s+s1}{\PYZsq{}}\PY{p}{)}
                    
                    \PY{c+c1}{\PYZsh{} Histogram subplot}
                    \PY{n}{ax\PYZus{}hist} \PY{o}{=} \PY{n}{fig}\PY{o}{.}\PY{n}{add\PYZus{}subplot}\PY{p}{(}\PY{n}{gs}\PY{p}{[}\PY{n}{plot\PYZus{}index}\PY{p}{,} \PY{l+m+mi}{1}\PY{p}{]}\PY{p}{)}
                    \PY{n}{ax\PYZus{}hist}\PY{o}{.}\PY{n}{hist}\PY{p}{(}\PY{n}{processed\PYZus{}image}\PY{o}{.}\PY{n}{ravel}\PY{p}{(}\PY{p}{)}\PY{p}{,} \PY{n}{bins}\PY{o}{=}\PY{l+m+mi}{256}\PY{p}{,} \PY{n+nb}{range}\PY{o}{=}\PY{p}{(}\PY{l+m+mi}{0}\PY{p}{,} \PY{l+m+mi}{255}\PY{p}{)}\PY{p}{,} \PY{n}{color}\PY{o}{=}\PY{l+s+s1}{\PYZsq{}}\PY{l+s+s1}{blue}\PY{l+s+s1}{\PYZsq{}}\PY{p}{,} \PY{n}{alpha}\PY{o}{=}\PY{l+m+mf}{0.7}\PY{p}{)}
                    \PY{n}{ax\PYZus{}hist}\PY{o}{.}\PY{n}{set\PYZus{}title}\PY{p}{(}\PY{l+s+sa}{f}\PY{l+s+s2}{\PYZdq{}}\PY{l+s+s2}{Histogram: Tile }\PY{l+s+si}{\PYZob{}}\PY{n}{size}\PY{l+s+si}{\PYZcb{}}\PY{l+s+s2}{, Clip }\PY{l+s+si}{\PYZob{}}\PY{n}{clip}\PY{l+s+si}{\PYZcb{}}\PY{l+s+s2}{\PYZdq{}}\PY{p}{,} \PY{n}{fontsize}\PY{o}{=}\PY{l+m+mi}{12}\PY{p}{)}
                    \PY{n}{ax\PYZus{}hist}\PY{o}{.}\PY{n}{set\PYZus{}xlabel}\PY{p}{(}\PY{l+s+s1}{\PYZsq{}}\PY{l+s+s1}{Pixel Intensity}\PY{l+s+s1}{\PYZsq{}}\PY{p}{)}
                    \PY{n}{ax\PYZus{}hist}\PY{o}{.}\PY{n}{set\PYZus{}ylabel}\PY{p}{(}\PY{l+s+s1}{\PYZsq{}}\PY{l+s+s1}{Frequency}\PY{l+s+s1}{\PYZsq{}}\PY{p}{)}
                    \PY{n}{ax\PYZus{}hist}\PY{o}{.}\PY{n}{grid}\PY{p}{(}\PY{k+kc}{True}\PY{p}{,} \PY{n}{alpha}\PY{o}{=}\PY{l+m+mf}{0.3}\PY{p}{)}
                    
                    \PY{n}{plot\PYZus{}index} \PY{o}{+}\PY{o}{=} \PY{l+m+mi}{1}
            
            \PY{n}{plt}\PY{o}{.}\PY{n}{suptitle}\PY{p}{(}\PY{l+s+sa}{f}\PY{l+s+s2}{\PYZdq{}}\PY{l+s+s2}{CLAHE Analysis for }\PY{l+s+si}{\PYZob{}}\PY{n}{img\PYZus{}name}\PY{l+s+si}{\PYZcb{}}\PY{l+s+s2}{\PYZdq{}}\PY{p}{,} \PY{n}{fontsize}\PY{o}{=}\PY{l+m+mi}{16}\PY{p}{,} \PY{n}{fontweight}\PY{o}{=}\PY{l+s+s1}{\PYZsq{}}\PY{l+s+s1}{bold}\PY{l+s+s1}{\PYZsq{}}\PY{p}{,} \PY{n}{y}\PY{o}{=}\PY{l+m+mf}{0.98}\PY{p}{)}
            \PY{n}{plt}\PY{o}{.}\PY{n}{tight\PYZus{}layout}\PY{p}{(}\PY{p}{)}
            \PY{n}{plt}\PY{o}{.}\PY{n}{show}\PY{p}{(}\PY{p}{)}
            
            \PY{c+c1}{\PYZsh{} Statistical analysis}
            \PY{n+nb}{print}\PY{p}{(}\PY{l+s+sa}{f}\PY{l+s+s2}{\PYZdq{}}\PY{l+s+se}{\PYZbs{}n}\PY{l+s+si}{\PYZob{}}\PY{n}{img\PYZus{}name}\PY{l+s+si}{\PYZcb{}}\PY{l+s+s2}{ \PYZhy{} Statistical Analysis:}\PY{l+s+s2}{\PYZdq{}}\PY{p}{)}
            \PY{n+nb}{print}\PY{p}{(}\PY{l+s+sa}{f}\PY{l+s+s2}{\PYZdq{}}\PY{l+s+s2}{Original \PYZhy{} Mean: }\PY{l+s+si}{\PYZob{}}\PY{n}{img\PYZus{}gray}\PY{o}{.}\PY{n}{mean}\PY{p}{(}\PY{p}{)}\PY{l+s+si}{:}\PY{l+s+s2}{.2f}\PY{l+s+si}{\PYZcb{}}\PY{l+s+s2}{, Std: }\PY{l+s+si}{\PYZob{}}\PY{n}{img\PYZus{}gray}\PY{o}{.}\PY{n}{std}\PY{p}{(}\PY{p}{)}\PY{l+s+si}{:}\PY{l+s+s2}{.2f}\PY{l+s+si}{\PYZcb{}}\PY{l+s+s2}{\PYZdq{}}\PY{p}{)}
            
            \PY{c+c1}{\PYZsh{} Test different parameters and show best results}
            \PY{n}{best\PYZus{}contrast} \PY{o}{=} \PY{l+m+mi}{0}
            \PY{n}{best\PYZus{}params} \PY{o}{=} \PY{k+kc}{None}
            \PY{n}{best\PYZus{}image} \PY{o}{=} \PY{k+kc}{None}
            
            \PY{k}{for} \PY{n}{size} \PY{o+ow}{in} \PY{n}{sizes}\PY{p}{:}
                \PY{k}{for} \PY{n}{clip} \PY{o+ow}{in} \PY{n}{clip\PYZus{}limits}\PY{p}{:}
                    \PY{n}{test\PYZus{}img} \PY{o}{=} \PY{n}{apply\PYZus{}clahe}\PY{p}{(}\PY{n}{img\PYZus{}gray}\PY{p}{,} \PY{n}{clip\PYZus{}limit}\PY{o}{=}\PY{n}{clip}\PY{p}{,} \PY{n}{tile\PYZus{}grid\PYZus{}size}\PY{o}{=}\PY{n}{size}\PY{p}{)}
                    \PY{n}{contrast} \PY{o}{=} \PY{n}{test\PYZus{}img}\PY{o}{.}\PY{n}{std}\PY{p}{(}\PY{p}{)}  \PY{c+c1}{\PYZsh{} Standard deviation as contrast measure}
                    \PY{k}{if} \PY{n}{contrast} \PY{o}{\PYZgt{}} \PY{n}{best\PYZus{}contrast}\PY{p}{:}
                        \PY{n}{best\PYZus{}contrast} \PY{o}{=} \PY{n}{contrast}
                        \PY{n}{best\PYZus{}params} \PY{o}{=} \PY{p}{(}\PY{n}{size}\PY{p}{,} \PY{n}{clip}\PY{p}{)}
                        \PY{n}{best\PYZus{}image} \PY{o}{=} \PY{n}{test\PYZus{}img}
            
            \PY{n+nb}{print}\PY{p}{(}\PY{l+s+sa}{f}\PY{l+s+s2}{\PYZdq{}}\PY{l+s+s2}{Best CLAHE \PYZhy{} Tile: }\PY{l+s+si}{\PYZob{}}\PY{n}{best\PYZus{}params}\PY{p}{[}\PY{l+m+mi}{0}\PY{p}{]}\PY{l+s+si}{\PYZcb{}}\PY{l+s+s2}{, Clip: }\PY{l+s+si}{\PYZob{}}\PY{n}{best\PYZus{}params}\PY{p}{[}\PY{l+m+mi}{1}\PY{p}{]}\PY{l+s+si}{\PYZcb{}}\PY{l+s+s2}{\PYZdq{}}\PY{p}{)}
            \PY{n+nb}{print}\PY{p}{(}\PY{l+s+sa}{f}\PY{l+s+s2}{\PYZdq{}}\PY{l+s+s2}{Best CLAHE \PYZhy{} Mean: }\PY{l+s+si}{\PYZob{}}\PY{n}{best\PYZus{}image}\PY{o}{.}\PY{n}{mean}\PY{p}{(}\PY{p}{)}\PY{l+s+si}{:}\PY{l+s+s2}{.2f}\PY{l+s+si}{\PYZcb{}}\PY{l+s+s2}{, Std: }\PY{l+s+si}{\PYZob{}}\PY{n}{best\PYZus{}image}\PY{o}{.}\PY{n}{std}\PY{p}{(}\PY{p}{)}\PY{l+s+si}{:}\PY{l+s+s2}{.2f}\PY{l+s+si}{\PYZcb{}}\PY{l+s+s2}{\PYZdq{}}\PY{p}{)}
            \PY{n+nb}{print}\PY{p}{(}\PY{l+s+sa}{f}\PY{l+s+s2}{\PYZdq{}}\PY{l+s+s2}{Contrast Improvement: }\PY{l+s+si}{\PYZob{}}\PY{p}{(}\PY{p}{(}\PY{n}{best\PYZus{}contrast}\PY{+w}{ }\PY{o}{\PYZhy{}}\PY{+w}{ }\PY{n}{img\PYZus{}gray}\PY{o}{.}\PY{n}{std}\PY{p}{(}\PY{p}{)}\PY{p}{)}\PY{+w}{ }\PY{o}{/}\PY{+w}{ }\PY{n}{img\PYZus{}gray}\PY{o}{.}\PY{n}{std}\PY{p}{(}\PY{p}{)}\PY{+w}{ }\PY{o}{*}\PY{+w}{ }\PY{l+m+mi}{100}\PY{p}{)}\PY{l+s+si}{:}\PY{l+s+s2}{.1f}\PY{l+s+si}{\PYZcb{}}\PY{l+s+s2}{\PYZpc{}}\PY{l+s+s2}{\PYZdq{}}\PY{p}{)}

    \PY{k}{except} \PY{n+ne}{Exception} \PY{k}{as} \PY{n}{e}\PY{p}{:}
        \PY{n+nb}{print}\PY{p}{(}\PY{l+s+sa}{f}\PY{l+s+s2}{\PYZdq{}}\PY{l+s+s2}{An unexpected error occurred: }\PY{l+s+si}{\PYZob{}}\PY{n}{e}\PY{l+s+si}{\PYZcb{}}\PY{l+s+s2}{\PYZdq{}}\PY{p}{)}
\end{Verbatim}
\end{tcolorbox}

    \begin{Verbatim}[commandchars=\\\{\}]

Processing Image 1{\ldots}
    \end{Verbatim}

    \begin{Verbatim}[commandchars=\\\{\}]
/var/folders/fc/52j9sv\_132gdd2pvx3lpbt180000gn/T/ipykernel\_19281/3437559743.py:6
2: UserWarning: This figure includes Axes that are not compatible with
tight\_layout, so results might be incorrect.
  plt.tight\_layout()
    \end{Verbatim}

    \begin{center}
    \adjustimage{max size={0.9\linewidth}{0.9\paperheight}}{output_10_2.png}
    \end{center}
    { \hspace*{\fill} \\}
    
    \begin{Verbatim}[commandchars=\\\{\}]

Image 1 - Statistical Analysis:
Original - Mean: 110.41, Std: 62.25
Best CLAHE - Tile: (4, 4), Clip: 2.0
Best CLAHE - Mean: 123.45, Std: 71.19
Contrast Improvement: 14.4\%

Processing Image 2{\ldots}
    \end{Verbatim}

    \begin{center}
    \adjustimage{max size={0.9\linewidth}{0.9\paperheight}}{output_10_4.png}
    \end{center}
    { \hspace*{\fill} \\}
    
    \begin{Verbatim}[commandchars=\\\{\}]

Image 2 - Statistical Analysis:
Original - Mean: 82.35, Std: 41.08
Best CLAHE - Tile: (4, 4), Clip: 4.0
Best CLAHE - Mean: 121.63, Std: 63.24
Contrast Improvement: 53.9\%

Processing Image 3{\ldots}
    \end{Verbatim}

    \begin{center}
    \adjustimage{max size={0.9\linewidth}{0.9\paperheight}}{output_10_6.png}
    \end{center}
    { \hspace*{\fill} \\}
    
    \begin{Verbatim}[commandchars=\\\{\}]

Image 3 - Statistical Analysis:
Original - Mean: 58.41, Std: 69.66
Best CLAHE - Tile: (4, 4), Clip: 4.0
Best CLAHE - Mean: 73.67, Std: 75.22
Contrast Improvement: 8.0\%
    \end{Verbatim}

    \hypertarget{how-to-improve-the-contrast-between-blocks}{%
\subsection{How to Improve the Contrast Between
Blocks}\label{how-to-improve-the-contrast-between-blocks}}

The primary drawback of a simple tile-based histogram equalization (AHE)
is the potential for \textbf{noticeable differences in contrast between
blocks}, resulting in a grid-like artifact. The implementation we used,
\textbf{CLAHE}, already incorporates the key improvements to solve this
issue:

\begin{enumerate}
\def\labelenumi{\arabic{enumi}.}
\item
  \textbf{Contrast Limiting}: By using a \texttt{clip\_limit}, CLAHE
  prevents any single histogram bin from having an extremely high count.
  This stops the algorithm from aggressively amplifying contrast in
  uniform regions (like a clear sky), which would otherwise lead to
  noise amplification and sharp contrast differences between tiles.
\item
  \textbf{Bilinear Interpolation}: This is the most critical feature.
  Instead of simply applying the equalization function for each tile and
  then putting the image back together, CLAHE calculates the final pixel
  value by \textbf{interpolating} the results from the four surrounding
  tiles. This creates a smooth, continuous transition across tile
  boundaries, effectively eliminating the grid artifacts.
\end{enumerate}

    \hypertarget{clahe-best-practices-and-usage-guidelines}{%
\subsection{CLAHE Best Practices and Usage
Guidelines}\label{clahe-best-practices-and-usage-guidelines}}

\hypertarget{when-to-use-clahe}{%
\subsubsection{When to Use CLAHE:}\label{when-to-use-clahe}}

\begin{enumerate}
\def\labelenumi{\arabic{enumi}.}
\tightlist
\item
  \textbf{Low-contrast images}: When global histogram equalization fails
  to reveal details in both bright and dark areas simultaneously.
\item
  \textbf{Medical imaging}: Enhances X-rays, CT scans, and MRI images
  where local contrast is crucial for diagnosis.
\item
  \textbf{Satellite imagery}: Improves contrast in aerial photos with
  varying lighting conditions.
\item
  \textbf{Photography}: Can be used to create an HDR-like effect,
  bringing out details in shadows and highlights from a single exposure.
\item
  \textbf{Computer vision preprocessing}: An excellent preprocessing
  step to enhance images before applying feature detection, object
  recognition, or segmentation algorithms.
\end{enumerate}

\hypertarget{parameter-selection-guide}{%
\subsubsection{Parameter Selection
Guide:}\label{parameter-selection-guide}}

\hypertarget{clip-limit-contrast-limiting}{%
\paragraph{Clip Limit (Contrast
Limiting):}\label{clip-limit-contrast-limiting}}

\begin{itemize}
\tightlist
\item
  \textbf{1.0-2.0}: \textbf{Conservative}, preserves a more natural look
  while providing a subtle enhancement.
\item
  \textbf{2.0-3.0}: A \textbf{balanced} enhancement, often the
  recommended starting point.
\item
  \textbf{3.0-4.0}: \textbf{Aggressive} enhancement; may introduce
  artifacts but can be useful for very low-contrast images.
\item
  \textbf{\textgreater4.0}: Usually too aggressive, with a high risk of
  amplifying noise and creating unnatural results.
\end{itemize}

\hypertarget{tile-grid-size}{%
\paragraph{Tile Grid Size:}\label{tile-grid-size}}

\begin{itemize}
\tightlist
\item
  \textbf{(4,4)}: Provides \textbf{very local adaptation} and is good
  for revealing fine details, but may create more visible artifacts if
  the clip limit is high.
\item
  \textbf{(8,8)}: Offers a \textbf{good balance} between local and
  global adaptation and is the recommended starting point.
\item
  \textbf{(16,16)}: Provides a \textbf{smoother} result with more global
  adaptation, similar to regular histogram equalization but with more
  local control.
\item
  \textbf{(32,32)}: Very global; the effect will be very similar to
  regular histogram equalization, losing the benefits of a tiled
  approach.
\end{itemize}

\hypertarget{best-practices}{%
\subsubsection{Best Practices:}\label{best-practices}}

\begin{enumerate}
\def\labelenumi{\arabic{enumi}.}
\tightlist
\item
  \textbf{Start with (8,8) tiles and a clip limit of 2.0.}
\item
  \textbf{Adjust the clip limit first, then the tile size}, to find the
  optimal balance between contrast enhancement and noise.
\item
  \textbf{Be mindful of over-enhancement artifacts}, which can make the
  image look artificial or ``grainy.''
\item
  \textbf{Use statistical measures} like standard deviation to
  objectively quantify the contrast improvement.
\end{enumerate}

    \hypertarget{exercise-2}{%
\subsubsection{Exercise 2}\label{exercise-2}}

Research and implement a simple method for Sliding Window Adaptive
Histogram Equalization (SWAHE), and compare its results on several
images of different types.

\textbf{Sliding Window Adaptive Histogram Equalization (SWAHE)} 

SWAHE applies histogram equalization using a sliding window centered on each pixel. This method adapts to local intensity variations and avoids block artifacts.

\hypertarget{section}{%
\section{-------------------------------}\label{section}}

    \begin{tcolorbox}[breakable, size=fbox, boxrule=1pt, pad at break*=1mm,colback=cellbackground, colframe=cellborder]
\prompt{In}{incolor}{23}{\boxspacing}
\begin{Verbatim}[commandchars=\\\{\}]
\PY{c+c1}{\PYZsh{} 3. SWAHE DEFINITION (IMPROVED WITH CLIP LIMIT)}
\PY{c+c1}{\PYZsh{}WINDOW\PYZus{}SIZE = 51  \PYZsh{} Can adjust this value}
\PY{n}{WINDOW\PYZus{}SIZE} \PY{o}{=} \PY{l+m+mi}{51}
\PY{n}{CLIP\PYZus{}LIMIT} \PY{o}{=} \PY{l+m+mf}{0.02}  \PY{c+c1}{\PYZsh{} Adjust this value (e.g., 0.01 to 0.05 are common starting points)}

\PY{k}{def}\PY{+w}{ }\PY{n+nf}{swahe}\PY{p}{(}\PY{n}{image}\PY{p}{,} \PY{n}{window\PYZus{}size}\PY{p}{,} \PY{n}{clip\PYZus{}limit}\PY{o}{=}\PY{l+m+mf}{0.02}\PY{p}{)}\PY{p}{:}
    \PY{k}{if} \PY{n}{window\PYZus{}size} \PY{o}{\PYZpc{}} \PY{l+m+mi}{2} \PY{o}{==} \PY{l+m+mi}{0}\PY{p}{:}
        \PY{k}{raise} \PY{n+ne}{ValueError}\PY{p}{(}\PY{l+s+s2}{\PYZdq{}}\PY{l+s+s2}{Window size must be odd.}\PY{l+s+s2}{\PYZdq{}}\PY{p}{)}

    \PY{n}{h}\PY{p}{,} \PY{n}{w} \PY{o}{=} \PY{n}{image}\PY{o}{.}\PY{n}{shape}
    \PY{n}{output\PYZus{}image} \PY{o}{=} \PY{n}{np}\PY{o}{.}\PY{n}{zeros\PYZus{}like}\PY{p}{(}\PY{n}{image}\PY{p}{)}
    \PY{n}{pad\PYZus{}size} \PY{o}{=} \PY{n}{window\PYZus{}size} \PY{o}{/}\PY{o}{/} \PY{l+m+mi}{2}
    \PY{n}{padded\PYZus{}image} \PY{o}{=} \PY{n}{np}\PY{o}{.}\PY{n}{pad}\PY{p}{(}\PY{n}{image}\PY{p}{,} \PY{n}{pad\PYZus{}size}\PY{p}{,} \PY{n}{mode}\PY{o}{=}\PY{l+s+s1}{\PYZsq{}}\PY{l+s+s1}{reflect}\PY{l+s+s1}{\PYZsq{}}\PY{p}{)}

    \PY{n+nb}{print}\PY{p}{(}\PY{l+s+sa}{f}\PY{l+s+s2}{\PYZdq{}}\PY{l+s+s2}{Processing a }\PY{l+s+si}{\PYZob{}}\PY{n}{h}\PY{l+s+si}{\PYZcb{}}\PY{l+s+s2}{x}\PY{l+s+si}{\PYZob{}}\PY{n}{w}\PY{l+s+si}{\PYZcb{}}\PY{l+s+s2}{ image with a }\PY{l+s+si}{\PYZob{}}\PY{n}{window\PYZus{}size}\PY{l+s+si}{\PYZcb{}}\PY{l+s+s2}{x}\PY{l+s+si}{\PYZob{}}\PY{n}{window\PYZus{}size}\PY{l+s+si}{\PYZcb{}}\PY{l+s+s2}{ window and clip\PYZus{}limit=}\PY{l+s+si}{\PYZob{}}\PY{n}{clip\PYZus{}limit}\PY{l+s+si}{\PYZcb{}}\PY{l+s+s2}{...}\PY{l+s+s2}{\PYZdq{}}\PY{p}{)}

    \PY{k}{for} \PY{n}{y} \PY{o+ow}{in} \PY{n+nb}{range}\PY{p}{(}\PY{n}{h}\PY{p}{)}\PY{p}{:}
        \PY{k}{for} \PY{n}{x} \PY{o+ow}{in} \PY{n+nb}{range}\PY{p}{(}\PY{n}{w}\PY{p}{)}\PY{p}{:}
            \PY{n}{window} \PY{o}{=} \PY{n}{padded\PYZus{}image}\PY{p}{[}\PY{n}{y} \PY{p}{:} \PY{n}{y} \PY{o}{+} \PY{n}{window\PYZus{}size}\PY{p}{,} \PY{n}{x} \PY{p}{:} \PY{n}{x} \PY{o}{+} \PY{n}{window\PYZus{}size}\PY{p}{]}
            \PY{n}{center\PYZus{}pixel\PYZus{}value} \PY{o}{=} \PY{n}{image}\PY{p}{[}\PY{n}{y}\PY{p}{,} \PY{n}{x}\PY{p}{]}

            \PY{c+c1}{\PYZsh{} Calculate histogram}
            \PY{n}{hist}\PY{p}{,} \PY{n}{bins} \PY{o}{=} \PY{n}{np}\PY{o}{.}\PY{n}{histogram}\PY{p}{(}\PY{n}{window}\PY{o}{.}\PY{n}{flatten}\PY{p}{(}\PY{p}{)}\PY{p}{,} \PY{n}{bins}\PY{o}{=}\PY{l+m+mi}{256}\PY{p}{,} \PY{n+nb}{range}\PY{o}{=}\PY{p}{[}\PY{l+m+mi}{0}\PY{p}{,} \PY{l+m+mi}{256}\PY{p}{]}\PY{p}{)}

            \PY{c+c1}{\PYZsh{} \PYZhy{}\PYZhy{}\PYZhy{} Apply Contrast Limiting (CLAHE\PYZhy{}like behavior) \PYZhy{}\PYZhy{}\PYZhy{}}
            \PY{n}{num\PYZus{}pixels\PYZus{}in\PYZus{}window} \PY{o}{=} \PY{n}{window}\PY{o}{.}\PY{n}{size}
            \PY{n}{max\PYZus{}bin\PYZus{}height} \PY{o}{=} \PY{n}{num\PYZus{}pixels\PYZus{}in\PYZus{}window} \PY{o}{*} \PY{n}{clip\PYZus{}limit}

            \PY{c+c1}{\PYZsh{} Clip the histogram bins}
            \PY{n}{excess} \PY{o}{=} \PY{l+m+mi}{0}
            \PY{k}{for} \PY{n}{i} \PY{o+ow}{in} \PY{n+nb}{range}\PY{p}{(}\PY{n+nb}{len}\PY{p}{(}\PY{n}{hist}\PY{p}{)}\PY{p}{)}\PY{p}{:}
                \PY{k}{if} \PY{n}{hist}\PY{p}{[}\PY{n}{i}\PY{p}{]} \PY{o}{\PYZgt{}} \PY{n}{max\PYZus{}bin\PYZus{}height}\PY{p}{:}
                    \PY{n}{excess} \PY{o}{+}\PY{o}{=} \PY{n}{hist}\PY{p}{[}\PY{n}{i}\PY{p}{]} \PY{o}{\PYZhy{}} \PY{n}{max\PYZus{}bin\PYZus{}height}
                    \PY{n}{hist}\PY{p}{[}\PY{n}{i}\PY{p}{]} \PY{o}{=} \PY{n}{max\PYZus{}bin\PYZus{}height}

            \PY{c+c1}{\PYZsh{} Redistribute the excess}
            \PY{n}{redistribute\PYZus{}per\PYZus{}bin} \PY{o}{=} \PY{n}{excess} \PY{o}{/} \PY{l+m+mi}{256}
            \PY{k}{for} \PY{n}{i} \PY{o+ow}{in} \PY{n+nb}{range}\PY{p}{(}\PY{n+nb}{len}\PY{p}{(}\PY{n}{hist}\PY{p}{)}\PY{p}{)}\PY{p}{:}
                \PY{n}{hist}\PY{p}{[}\PY{n}{i}\PY{p}{]} \PY{o}{+}\PY{o}{=} \PY{n}{redistribute\PYZus{}per\PYZus{}bin}
            \PY{c+c1}{\PYZsh{} \PYZhy{}\PYZhy{}\PYZhy{} End Contrast Limiting \PYZhy{}\PYZhy{}\PYZhy{}}

            \PY{c+c1}{\PYZsh{} Calculate CDF}
            \PY{n}{cdf} \PY{o}{=} \PY{n}{hist}\PY{o}{.}\PY{n}{cumsum}\PY{p}{(}\PY{p}{)}

            \PY{k}{if} \PY{n}{cdf}\PY{p}{[}\PY{o}{\PYZhy{}}\PY{l+m+mi}{1}\PY{p}{]} \PY{o}{==} \PY{l+m+mi}{0}\PY{p}{:}
                \PY{n}{output\PYZus{}image}\PY{p}{[}\PY{n}{y}\PY{p}{,} \PY{n}{x}\PY{p}{]} \PY{o}{=} \PY{l+m+mi}{0}
                \PY{k}{continue}

            \PY{c+c1}{\PYZsh{} Normalize CDF to 0\PYZhy{}255}
            \PY{n}{cdf\PYZus{}normalized} \PY{o}{=} \PY{p}{(}\PY{n}{cdf} \PY{o}{*} \PY{l+m+mi}{255} \PY{o}{/} \PY{n}{cdf}\PY{p}{[}\PY{o}{\PYZhy{}}\PY{l+m+mi}{1}\PY{p}{]}\PY{p}{)}\PY{o}{.}\PY{n}{astype}\PY{p}{(}\PY{n}{np}\PY{o}{.}\PY{n}{uint8}\PY{p}{)}

            \PY{c+c1}{\PYZsh{} Map the center pixel\PYZsq{}s value}
            \PY{n}{output\PYZus{}image}\PY{p}{[}\PY{n}{y}\PY{p}{,} \PY{n}{x}\PY{p}{]} \PY{o}{=} \PY{n}{cdf\PYZus{}normalized}\PY{p}{[}\PY{n}{center\PYZus{}pixel\PYZus{}value}\PY{p}{]}

    \PY{n+nb}{print}\PY{p}{(}\PY{l+s+s2}{\PYZdq{}}\PY{l+s+s2}{Processing complete!}\PY{l+s+s2}{\PYZdq{}}\PY{p}{)}
    \PY{k}{return} \PY{n}{output\PYZus{}image}

\PY{c+c1}{\PYZsh{} \PYZhy{}\PYZhy{}\PYZhy{} Processing loop for img1, img2, img3 \PYZhy{}\PYZhy{}\PYZhy{}}
\PY{c+c1}{\PYZsh{} Images already imported as img1, img2, img3 (see earlier code)}
\PY{n}{image\PYZus{}objs} \PY{o}{=} \PY{p}{[}\PY{n}{img10}\PY{p}{]}
\PY{n}{image\PYZus{}names} \PY{o}{=} \PY{p}{[}\PY{l+s+s1}{\PYZsq{}}\PY{l+s+s1}{img10}\PY{l+s+s1}{\PYZsq{}}\PY{p}{]}

\PY{k}{for} \PY{n}{image\PYZus{}obj}\PY{p}{,} \PY{n}{image\PYZus{}name} \PY{o+ow}{in} \PY{n+nb}{zip}\PY{p}{(}\PY{n}{image\PYZus{}objs}\PY{p}{,} \PY{n}{image\PYZus{}names}\PY{p}{)}\PY{p}{:}
    \PY{k}{if} \PY{n}{image\PYZus{}obj} \PY{o+ow}{is} \PY{k+kc}{None}\PY{p}{:}
        \PY{n+nb}{print}\PY{p}{(}\PY{l+s+sa}{f}\PY{l+s+s2}{\PYZdq{}}\PY{l+s+se}{\PYZbs{}n}\PY{l+s+s2}{WARNING: Image object }\PY{l+s+s2}{\PYZsq{}}\PY{l+s+si}{\PYZob{}}\PY{n}{image\PYZus{}name}\PY{l+s+si}{\PYZcb{}}\PY{l+s+s2}{\PYZsq{}}\PY{l+s+s2}{ is None. Skipping.}\PY{l+s+s2}{\PYZdq{}}\PY{p}{)}
        \PY{k}{continue}

    \PY{n+nb}{print}\PY{p}{(}\PY{l+s+sa}{f}\PY{l+s+s2}{\PYZdq{}}\PY{l+s+se}{\PYZbs{}n}\PY{l+s+s2}{Processing: }\PY{l+s+si}{\PYZob{}}\PY{n}{image\PYZus{}name}\PY{l+s+si}{\PYZcb{}}\PY{l+s+s2}{\PYZdq{}}\PY{p}{)}

    \PY{c+c1}{\PYZsh{} If image is RGB, convert to grayscale}
    \PY{k}{if} \PY{n}{image\PYZus{}obj}\PY{o}{.}\PY{n}{ndim} \PY{o}{==} \PY{l+m+mi}{3}\PY{p}{:}
        \PY{c+c1}{\PYZsh{} Handle alpha channel if present}
        \PY{k}{if} \PY{n}{image\PYZus{}obj}\PY{o}{.}\PY{n}{shape}\PY{p}{[}\PY{l+m+mi}{2}\PY{p}{]} \PY{o}{==} \PY{l+m+mi}{4}\PY{p}{:}
            \PY{n}{gray\PYZus{}image} \PY{o}{=} \PY{n}{cv2}\PY{o}{.}\PY{n}{cvtColor}\PY{p}{(}\PY{p}{(}\PY{n}{image\PYZus{}obj}\PY{p}{[}\PY{o}{.}\PY{o}{.}\PY{o}{.}\PY{p}{,} \PY{p}{:}\PY{l+m+mi}{3}\PY{p}{]} \PY{o}{*} \PY{l+m+mi}{255}\PY{p}{)}\PY{o}{.}\PY{n}{astype}\PY{p}{(}\PY{n}{np}\PY{o}{.}\PY{n}{uint8}\PY{p}{)}\PY{p}{,} \PY{n}{cv2}\PY{o}{.}\PY{n}{COLOR\PYZus{}RGB2GRAY}\PY{p}{)}
        \PY{k}{else}\PY{p}{:}
            \PY{n}{gray\PYZus{}image} \PY{o}{=} \PY{n}{cv2}\PY{o}{.}\PY{n}{cvtColor}\PY{p}{(}\PY{p}{(}\PY{n}{image\PYZus{}obj} \PY{o}{*} \PY{l+m+mi}{255}\PY{p}{)}\PY{o}{.}\PY{n}{astype}\PY{p}{(}\PY{n}{np}\PY{o}{.}\PY{n}{uint8}\PY{p}{)}\PY{p}{,} \PY{n}{cv2}\PY{o}{.}\PY{n}{COLOR\PYZus{}RGB2GRAY}\PY{p}{)}
    \PY{k}{else}\PY{p}{:}
        \PY{c+c1}{\PYZsh{} Already grayscale, but ensure uint8}
        \PY{k}{if} \PY{n}{image\PYZus{}obj}\PY{o}{.}\PY{n}{dtype} \PY{o}{!=} \PY{n}{np}\PY{o}{.}\PY{n}{uint8}\PY{p}{:}
            \PY{c+c1}{\PYZsh{} Assume float in 0\PYZhy{}1}
            \PY{n}{gray\PYZus{}image} \PY{o}{=} \PY{p}{(}\PY{n}{image\PYZus{}obj} \PY{o}{*} \PY{l+m+mi}{255}\PY{p}{)}\PY{o}{.}\PY{n}{astype}\PY{p}{(}\PY{n}{np}\PY{o}{.}\PY{n}{uint8}\PY{p}{)}
        \PY{k}{else}\PY{p}{:}
            \PY{n}{gray\PYZus{}image} \PY{o}{=} \PY{n}{image\PYZus{}obj}

    \PY{c+c1}{\PYZsh{} Apply SWAHE with the improved function and clip\PYZus{}limit}
    \PY{n}{swahe\PYZus{}result} \PY{o}{=} \PY{n}{swahe}\PY{p}{(}\PY{n}{gray\PYZus{}image}\PY{p}{,} \PY{n}{window\PYZus{}size}\PY{o}{=}\PY{n}{WINDOW\PYZus{}SIZE}\PY{p}{,} \PY{n}{clip\PYZus{}limit}\PY{o}{=}\PY{n}{CLIP\PYZus{}LIMIT}\PY{p}{)}

    \PY{c+c1}{\PYZsh{} Display the results}
    \PY{n}{fig}\PY{p}{,} \PY{n}{axes} \PY{o}{=} \PY{n}{plt}\PY{o}{.}\PY{n}{subplots}\PY{p}{(}\PY{l+m+mi}{1}\PY{p}{,} \PY{l+m+mi}{2}\PY{p}{,} \PY{n}{figsize}\PY{o}{=}\PY{p}{(}\PY{l+m+mi}{16}\PY{p}{,} \PY{l+m+mi}{8}\PY{p}{)}\PY{p}{)}

    \PY{n}{axes}\PY{p}{[}\PY{l+m+mi}{0}\PY{p}{]}\PY{o}{.}\PY{n}{imshow}\PY{p}{(}\PY{n}{gray\PYZus{}image}\PY{p}{,} \PY{n}{cmap}\PY{o}{=}\PY{l+s+s1}{\PYZsq{}}\PY{l+s+s1}{gray}\PY{l+s+s1}{\PYZsq{}}\PY{p}{)}
    \PY{n}{axes}\PY{p}{[}\PY{l+m+mi}{0}\PY{p}{]}\PY{o}{.}\PY{n}{set\PYZus{}title}\PY{p}{(}\PY{l+s+s1}{\PYZsq{}}\PY{l+s+s1}{Original (Grayscale)}\PY{l+s+s1}{\PYZsq{}}\PY{p}{)}
    \PY{n}{axes}\PY{p}{[}\PY{l+m+mi}{0}\PY{p}{]}\PY{o}{.}\PY{n}{axis}\PY{p}{(}\PY{l+s+s1}{\PYZsq{}}\PY{l+s+s1}{off}\PY{l+s+s1}{\PYZsq{}}\PY{p}{)}

    \PY{n}{axes}\PY{p}{[}\PY{l+m+mi}{1}\PY{p}{]}\PY{o}{.}\PY{n}{imshow}\PY{p}{(}\PY{n}{swahe\PYZus{}result}\PY{p}{,} \PY{n}{cmap}\PY{o}{=}\PY{l+s+s1}{\PYZsq{}}\PY{l+s+s1}{gray}\PY{l+s+s1}{\PYZsq{}}\PY{p}{)}
    \PY{n}{axes}\PY{p}{[}\PY{l+m+mi}{1}\PY{p}{]}\PY{o}{.}\PY{n}{set\PYZus{}title}\PY{p}{(}\PY{l+s+sa}{f}\PY{l+s+s1}{\PYZsq{}}\PY{l+s+s1}{SWAHE Result (Window }\PY{l+s+si}{\PYZob{}}\PY{n}{WINDOW\PYZus{}SIZE}\PY{l+s+si}{\PYZcb{}}\PY{l+s+s1}{x}\PY{l+s+si}{\PYZob{}}\PY{n}{WINDOW\PYZus{}SIZE}\PY{l+s+si}{\PYZcb{}}\PY{l+s+s1}{, Clip }\PY{l+s+si}{\PYZob{}}\PY{n}{CLIP\PYZus{}LIMIT}\PY{l+s+si}{\PYZcb{}}\PY{l+s+s1}{)}\PY{l+s+s1}{\PYZsq{}}\PY{p}{)}
    \PY{n}{axes}\PY{p}{[}\PY{l+m+mi}{1}\PY{p}{]}\PY{o}{.}\PY{n}{axis}\PY{p}{(}\PY{l+s+s1}{\PYZsq{}}\PY{l+s+s1}{off}\PY{l+s+s1}{\PYZsq{}}\PY{p}{)}

    \PY{n}{fig}\PY{o}{.}\PY{n}{suptitle}\PY{p}{(}\PY{n}{image\PYZus{}name}\PY{p}{,} \PY{n}{fontsize}\PY{o}{=}\PY{l+m+mi}{16}\PY{p}{)}
    \PY{n}{plt}\PY{o}{.}\PY{n}{show}\PY{p}{(}\PY{p}{)}
\end{Verbatim}
\end{tcolorbox}

    \begin{Verbatim}[commandchars=\\\{\}]

Processing: img10
Processing a 256x256 image with a 51x51 window and clip\_limit=0.02{\ldots}
Processing complete!
    \end{Verbatim}

    \begin{center}
    \adjustimage{max size={0.9\linewidth}{0.9\paperheight}}{output_14_1.png}
    \end{center}

\textbf{Conclusion}  

This practice successfully demonstrated the application of Sliding Window Adaptive Histogram Equalization (SWAHE) for enhancing image contrast. By processing images with a sliding window, the SWAHE algorithm effectively adapts the histogram equalization to local image characteristics, resulting in improved detail visibility compared to global histogram equalization methods.

The visual comparison between the original grayscale images and their SWAHE-processed counterparts clearly illustrated the enhancement achieved, particularly in areas with varying light conditions or subtle details. The use of a configurable window size allowed for exploration of its impact on the resulting image enhancement.

It's worth noting that while effective, the SWAHE process, especially with larger images or window sizes, can be computationally intensive and result in longer processing times compared to simpler methods.

This exercise highlights the utility of adaptive histogram equalization techniques like SWAHE in computer vision tasks where local contrast enhancement is crucial for subsequent analysis or visualization.
    { \hspace*{\fill} \\}
    
    \hypertarget{exercise-3}{%
\subsubsection{Exercise 3}\label{exercise-3}}

Research how the CLAHE algorithm (Contrast Limited Adaptive Histogram
Equalization) works and implement it (you may use the OpenCV
implementation). Test it on different types of images and compare the
results with the basic histogram equalization method. Provide a brief
description of the method based on your literature review.

\hypertarget{section}{%
\section{-------------------------------}\label{section}}

    \hypertarget{objective}{%
\subsubsection{Objective}\label{objective}}

The goal of this exercise is to investigate the \textbf{Contrast Limited
Adaptive Histogram Equalization (CLAHE)} algorithm, describe its
functioning based on bibliographic research, and implement it using
OpenCV. Different images will be processed with both \textbf{basic
histogram equalization (HE)} and \textbf{CLAHE} in order to compare
their performance in terms of visual quality and contrast enhancement.
The results aim to highlight the advantages and limitations of CLAHE
over the classical method.

    \hypertarget{description-of-the-method}{%
\subsubsection{Description of the
Method}\label{description-of-the-method}}

\textbf{Contrast Limited Adaptive Histogram Equalization (CLAHE).}\\
CLAHE is an advanced technique for enhancing the contrast of digital
images. Unlike traditional histogram equalization, which applies the
same transformation to the entire image, CLAHE works locally: it divides
the image into smaller blocks called tiles and analyzes the distribution
of pixel intensities within each one. This localized approach allows the
method to adapt to variations in brightness and contrast across
different regions of the image, making subtle details more visible even
under uneven illumination. To avoid exaggerated contrast or noise in
uniform areas, CLAHE introduces a mechanism that limits how much each
intensity value can grow in the histogram. Any excess is redistributed
fairly across the intensity range, ensuring balance. Finally, because
each tile is processed independently, the algorithm blends the results
smoothly through interpolation, preventing abrupt edges between tiles.
In this way, CLAHE achieves a compromise: it enhances local contrast and
reveals fine structures while controlling noise and preserving visual
continuity across the image.

In the following section, this procedure will be described
mathematically, presenting the sequence of operations and the role of
each parameter through formal equations.

\textbf{Step 1. Local histogram per tile.}

For a tile of size:

\[
M \times N
\]

with gray levels:

\[
0,1,\dots,L-1,
\]

the histogram is defined as:

\[
h_i = \text{number of pixels with intensity level } i, \quad i \in [0,L-1].
\]

Where:

\[
M = \text{number of rows (height) of the tile,}
\]

\[
N = \text{number of columns (width) of the tile,}
\]

\[
L = \text{number of possible gray levels,}
\]

\[
h_i = \text{histogram bin count at gray level } i.
\]

\textbf{Step 2. Apply the clip limit.}

To prevent over-amplification of noise, each histogram bin is clipped
as:

\[
h'_i = \min(h_i, C).
\]

Where:

\[
h'_i = \text{clipped histogram bin count at level } i,
\]

\[
h_i = \text{original histogram bin count at level } i,
\]

\[
C = \text{clip limit (maximum allowed bin height).}
\]

\textbf{Step 3. Redistribute the excess.}

The number of counts exceeding the clip limit is:

\[
\Delta = \sum_{i=0}^{L-1} \max(0, h_i - C).
\]

Where:

\[
\Delta = \text{total number of clipped pixels,}
\]

\[
h_i = \text{original histogram bin count at level } i,
\]

\[
C = \text{clip limit.}
\]

After redistribution, the adjusted histogram satisfies:

\[
\sum_{i=0}^{L-1} h''_i = M \times N
\]

Where:

\[
h''_i = \text{histogram after clipping and redistribution.}
\]

\textbf{Step 4. Compute the mapping (CDF).}

The equalization mapping per tile is:

\[
s_k = \operatorname{round}\!\left( (L-1) \sum_{j=0}^{k} \frac{h''_j}{M \times N} \right).
\]

Where:

\[
s_k = \text{output mapped gray level for input } r_k,
\]

\[
r_k = \text{input gray level,}
\]

\[
h''_j = \text{redistributed histogram bin count at level } j,
\]

\[
M \times N = \text{total number of pixels in the tile.}
\]

\textbf{Step 5. Bilinear interpolation across tiles.}

To avoid block artifacts, mappings from neighboring tiles are
interpolated. For a pixel at normalized coordinates ((\alpha,\beta))
inside a 2×2 neighborhood:

\[
s = (1-\alpha)(1-\beta)\,s_{00} + \alpha(1-\beta)\,s_{10} + (1-\alpha)\beta\,s_{01} + \alpha\beta\,s_{11}.
\]

Where:

\[
\alpha, \beta \in [0,1] = \text{relative position of the pixel in the tile neighborhood,}
\]

\[
s_{00}, s_{10}, s_{01}, s_{11} = \text{mapped outputs from the four adjacent tiles.}
\]

\textbf{Parameter effects.}

\begin{itemize}
\tightlist
\item
  Tile size:
\end{itemize}

\[
M \times N
\]

Smaller values → more local adaptation, risk of blockiness. Larger
values → closer to global HE.

\begin{itemize}
\tightlist
\item
  Clip limit:
\end{itemize}

\[
C
\]

Higher values → stronger contrast, more noise. Lower values → less
noise, weaker enhancement.

\textbf{Summary.}\\
In essence, CLAHE follows a structured pipeline composed of five main
stages:

\begin{enumerate}
\def\labelenumi{\arabic{enumi}.}
\item
  \textbf{Divide the image into tiles.}\\
  The image is partitioned into non-overlapping regions so that contrast
  enhancement can adapt to local intensity variations rather than
  relying on a single global transformation.
\item
  \textbf{Compute local histograms.}\\
  Within each tile, a histogram of gray-level intensities is built,
  capturing the distribution of pixel values in that region.
\item
  \textbf{Clip bins and redistribute the excess.}\\
  To avoid excessive amplification of noise, each histogram bin is
  capped at a clip limit. Counts above this threshold are accumulated as
  ``excess'' and redistributed uniformly across all bins, ensuring that
  the total pixel count remains unchanged.
\item
  \textbf{Apply CDF-based mapping.}\\
  Using the adjusted histogram, a cumulative distribution function (CDF)
  is calculated. This function remaps the original pixel intensities to
  new values, stretching the local dynamic range and improving contrast.
\item
  \textbf{Interpolate across tiles.}\\
  Because each tile is processed independently, their borders could
  otherwise appear discontinuous. To prevent this, CLAHE applies
  bilinear interpolation between adjacent tiles, blending results
  smoothly and avoiding block artifacts.
\end{enumerate}

Through these steps, CLAHE establishes a balance between \textbf{local
contrast enhancement} and \textbf{noise suppression}. By restricting the
amplification of homogeneous regions while still enhancing edges and
fine structures, it manages to improve visibility in images where global
methods like HE often fail. This makes CLAHE particularly effective in
demanding applications such as \textbf{medical imaging} (CT scans, MRI,
X-rays), \textbf{low-light photography}, and \textbf{remote sensing},
where preserving subtle local details is critical. Its versatility and
robustness have positioned CLAHE as a standard choice whenever contrast
must be improved without compromising image quality.

\begin{center}\rule{0.5\linewidth}{0.5pt}\end{center}

    \textbf{Histogram Equalization (HE).}\\
Histogram Equalization is a global technique for image contrast
enhancement. Its goal is to redistribute pixel intensity values so that
the resulting histogram is approximately uniform, meaning that all
intensity levels are represented more evenly across the image. This
process stretches the most frequent intensity values over a wider range,
thereby increasing the global contrast. The method works by computing
the cumulative distribution function (CDF) of the entire image and using
it as a mapping function to remap input intensities into output values.
As a result, darker regions are brightened and lighter regions are
darkened, improving overall visibility. However, since the same mapping
is applied to the entire image, this approach may over-enhance noise and
is prone to washing out fine local details in regions with homogeneous
intensity distributions (Wang \& Yang, 2024; Ochoa Ruiz, 2025).

\begin{center}\rule{0.5\linewidth}{0.5pt}\end{center}

\textbf{Comparison.}\\
While both Histogram Equalization (HE) and Contrast Limited Adaptive
Histogram Equalization (CLAHE) aim to improve visibility through
histogram manipulation, they differ fundamentally in scope and
robustness. HE applies a \textbf{single global transformation}, which
can be effective in evenly distributed images but often produces
excessive brightness, loss of detail, or amplification of noise in
challenging conditions. CLAHE, in contrast, operates \textbf{locally on
small tiles} and applies a clipping mechanism that limits the
amplification of any intensity range. This local and controlled
adjustment allows CLAHE to reveal fine structures and subtle contrasts
while preventing noise over-enhancement. Consequently, CLAHE is
generally more reliable than HE for low-light images, low-resolution
content, or environments with high levels of noise, as it preserves
detail without introducing severe artifacts (Mohammed \& Isa, 2025; Soni
\& Waoo, 2025).

    \begin{center}\rule{0.5\linewidth}{0.5pt}\end{center}

\hypertarget{image-processing-with-clahe-and-he}{%
\subsubsection{Image Processing with CLAHE and
HE}\label{image-processing-with-clahe-and-he}}

\textbf{3.3 Image Processing with HE and CLAHE}\\
To evaluate the practical performance of both methods, we will process
three different test images that represent distinct visual conditions.
The first image corresponds to an indoor wrestling event captured under
\textbf{low-light conditions}, where details in the darker regions are
difficult to perceive. The second image shows a \textbf{set of colorful
flowers in daylight}, which provides a scenario with rich colors and
moderate contrast. The third image depicts a \textbf{wall with cracks
and subtle texture}, representing a low-contrast and nearly uniform
surface. By applying both Histogram Equalization (HE) and Contrast
Limited Adaptive Histogram Equalization (CLAHE) to these cases, it
becomes possible to analyze how each algorithm behaves when dealing with
dark environments, complex color distributions, and homogeneous regions.
This will allow us to highlight the advantages and limitations of the
two methods under diverse imaging conditions.

    \begin{tcolorbox}[breakable, size=fbox, boxrule=1pt, pad at break*=1mm,colback=cellbackground, colframe=cellborder]
\prompt{In}{incolor}{24}{\boxspacing}
\begin{Verbatim}[commandchars=\\\{\}]
\PY{c+c1}{\PYZsh{} ================================================================}
\PY{c+c1}{\PYZsh{} Loading Images (img7, img8, img9)}
\PY{c+c1}{\PYZsh{} \PYZhy{}\PYZhy{}\PYZhy{}\PYZhy{}\PYZhy{}\PYZhy{}\PYZhy{}\PYZhy{}\PYZhy{}\PYZhy{}\PYZhy{}\PYZhy{}\PYZhy{}\PYZhy{}\PYZhy{}\PYZhy{}\PYZhy{}\PYZhy{}\PYZhy{}\PYZhy{}\PYZhy{}\PYZhy{}\PYZhy{}\PYZhy{}\PYZhy{}\PYZhy{}\PYZhy{}\PYZhy{}\PYZhy{}\PYZhy{}\PYZhy{}\PYZhy{}\PYZhy{}\PYZhy{}\PYZhy{}\PYZhy{}\PYZhy{}\PYZhy{}\PYZhy{}\PYZhy{}\PYZhy{}\PYZhy{}\PYZhy{}\PYZhy{}\PYZhy{}\PYZhy{}\PYZhy{}\PYZhy{}}
\PY{c+c1}{\PYZsh{} En esta sección cargamos las tres imágenes seleccionadas}
\PY{c+c1}{\PYZsh{} para el procesamiento con HE y CLAHE.}
\PY{c+c1}{\PYZsh{} ================================================================}

\PY{c+c1}{\PYZsh{} Load images}
\PY{k}{try}\PY{p}{:}
    \PY{n}{img7} \PY{o}{=} \PY{n}{mpimg}\PY{o}{.}\PY{n}{imread}\PY{p}{(}\PY{l+s+s1}{\PYZsq{}}\PY{l+s+s1}{data/img7.JPEG}\PY{l+s+s1}{\PYZsq{}}\PY{p}{)}   \PY{c+c1}{\PYZsh{} Evento deportivo (low\PYZhy{}light)}
    \PY{n}{img8} \PY{o}{=} \PY{n}{mpimg}\PY{o}{.}\PY{n}{imread}\PY{p}{(}\PY{l+s+s1}{\PYZsq{}}\PY{l+s+s1}{data/img8.JPEG}\PY{l+s+s1}{\PYZsq{}}\PY{p}{)}   \PY{c+c1}{\PYZsh{} Flores (contraste y color)}
    \PY{n}{img9} \PY{o}{=} \PY{n}{mpimg}\PY{o}{.}\PY{n}{imread}\PY{p}{(}\PY{l+s+s1}{\PYZsq{}}\PY{l+s+s1}{data/img9.JPG}\PY{l+s+s1}{\PYZsq{}}\PY{p}{)}    \PY{c+c1}{\PYZsh{} Pared (bajo contraste homogéneo)}

    \PY{n+nb}{print}\PY{p}{(}\PY{l+s+s2}{\PYZdq{}}\PY{l+s+s2}{Images loaded successfully!}\PY{l+s+s2}{\PYZdq{}}\PY{p}{)}
    \PY{n+nb}{print}\PY{p}{(}\PY{l+s+s2}{\PYZdq{}}\PY{l+s+s2}{Image 7 shape:}\PY{l+s+s2}{\PYZdq{}}\PY{p}{,} \PY{n}{img7}\PY{o}{.}\PY{n}{shape}\PY{p}{)}
    \PY{n+nb}{print}\PY{p}{(}\PY{l+s+s2}{\PYZdq{}}\PY{l+s+s2}{Image 8 shape:}\PY{l+s+s2}{\PYZdq{}}\PY{p}{,} \PY{n}{img8}\PY{o}{.}\PY{n}{shape}\PY{p}{)}
    \PY{n+nb}{print}\PY{p}{(}\PY{l+s+s2}{\PYZdq{}}\PY{l+s+s2}{Image 9 shape:}\PY{l+s+s2}{\PYZdq{}}\PY{p}{,} \PY{n}{img9}\PY{o}{.}\PY{n}{shape}\PY{p}{)}
\PY{k}{except} \PY{n+ne}{FileNotFoundError} \PY{k}{as} \PY{n}{e}\PY{p}{:}
    \PY{n+nb}{print}\PY{p}{(}\PY{l+s+sa}{f}\PY{l+s+s2}{\PYZdq{}}\PY{l+s+s2}{Error loading images: }\PY{l+s+si}{\PYZob{}}\PY{n}{e}\PY{l+s+si}{\PYZcb{}}\PY{l+s+s2}{\PYZdq{}}\PY{p}{)}
    \PY{c+c1}{\PYZsh{} Handle gracefully if images are not found}
    \PY{n}{img7}\PY{p}{,} \PY{n}{img8}\PY{p}{,} \PY{n}{img9} \PY{o}{=} \PY{k+kc}{None}\PY{p}{,} \PY{k+kc}{None}\PY{p}{,} \PY{k+kc}{None}

\PY{c+c1}{\PYZsh{} Display the selected images}
\PY{k}{if} \PY{n}{img7} \PY{o+ow}{is} \PY{o+ow}{not} \PY{k+kc}{None} \PY{o+ow}{and} \PY{n}{img8} \PY{o+ow}{is} \PY{o+ow}{not} \PY{k+kc}{None} \PY{o+ow}{and} \PY{n}{img9} \PY{o+ow}{is} \PY{o+ow}{not} \PY{k+kc}{None}\PY{p}{:}
    \PY{n}{fig}\PY{p}{,} \PY{n}{axes} \PY{o}{=} \PY{n}{plt}\PY{o}{.}\PY{n}{subplots}\PY{p}{(}\PY{l+m+mi}{1}\PY{p}{,} \PY{l+m+mi}{3}\PY{p}{,} \PY{n}{figsize}\PY{o}{=}\PY{p}{(}\PY{l+m+mi}{15}\PY{p}{,} \PY{l+m+mi}{5}\PY{p}{)}\PY{p}{)}

    \PY{n}{axes}\PY{p}{[}\PY{l+m+mi}{0}\PY{p}{]}\PY{o}{.}\PY{n}{imshow}\PY{p}{(}\PY{n}{img7}\PY{p}{)}
    \PY{n}{axes}\PY{p}{[}\PY{l+m+mi}{0}\PY{p}{]}\PY{o}{.}\PY{n}{set\PYZus{}title}\PY{p}{(}\PY{l+s+s1}{\PYZsq{}}\PY{l+s+s1}{Image 7 — Low Light}\PY{l+s+s1}{\PYZsq{}}\PY{p}{)}
    \PY{n}{axes}\PY{p}{[}\PY{l+m+mi}{0}\PY{p}{]}\PY{o}{.}\PY{n}{axis}\PY{p}{(}\PY{l+s+s1}{\PYZsq{}}\PY{l+s+s1}{off}\PY{l+s+s1}{\PYZsq{}}\PY{p}{)}

    \PY{n}{axes}\PY{p}{[}\PY{l+m+mi}{1}\PY{p}{]}\PY{o}{.}\PY{n}{imshow}\PY{p}{(}\PY{n}{img8}\PY{p}{)}
    \PY{n}{axes}\PY{p}{[}\PY{l+m+mi}{1}\PY{p}{]}\PY{o}{.}\PY{n}{set\PYZus{}title}\PY{p}{(}\PY{l+s+s1}{\PYZsq{}}\PY{l+s+s1}{Image 8 — Flowers}\PY{l+s+s1}{\PYZsq{}}\PY{p}{)}
    \PY{n}{axes}\PY{p}{[}\PY{l+m+mi}{1}\PY{p}{]}\PY{o}{.}\PY{n}{axis}\PY{p}{(}\PY{l+s+s1}{\PYZsq{}}\PY{l+s+s1}{off}\PY{l+s+s1}{\PYZsq{}}\PY{p}{)}

    \PY{n}{axes}\PY{p}{[}\PY{l+m+mi}{2}\PY{p}{]}\PY{o}{.}\PY{n}{imshow}\PY{p}{(}\PY{n}{img9}\PY{p}{)}
    \PY{n}{axes}\PY{p}{[}\PY{l+m+mi}{2}\PY{p}{]}\PY{o}{.}\PY{n}{set\PYZus{}title}\PY{p}{(}\PY{l+s+s1}{\PYZsq{}}\PY{l+s+s1}{Image 9 — Wall}\PY{l+s+s1}{\PYZsq{}}\PY{p}{)}
    \PY{n}{axes}\PY{p}{[}\PY{l+m+mi}{2}\PY{p}{]}\PY{o}{.}\PY{n}{axis}\PY{p}{(}\PY{l+s+s1}{\PYZsq{}}\PY{l+s+s1}{off}\PY{l+s+s1}{\PYZsq{}}\PY{p}{)}

    \PY{n}{plt}\PY{o}{.}\PY{n}{tight\PYZus{}layout}\PY{p}{(}\PY{p}{)}
    \PY{n}{plt}\PY{o}{.}\PY{n}{show}\PY{p}{(}\PY{p}{)}
\end{Verbatim}
\end{tcolorbox}

    \begin{Verbatim}[commandchars=\\\{\}]
Images loaded successfully!
Image 7 shape: (3024, 3024, 3)
Image 8 shape: (4032, 3024, 3)
Image 9 shape: (1826, 2014, 3)
    \end{Verbatim}

    \begin{center}
    \adjustimage{max size={0.9\linewidth}{0.9\paperheight}}{output_20_1.png}
    \end{center}
    { \hspace*{\fill} \\}
    
    \begin{tcolorbox}[breakable, size=fbox, boxrule=1pt, pad at break*=1mm,colback=cellbackground, colframe=cellborder]
\prompt{In}{incolor}{25}{\boxspacing}
\begin{Verbatim}[commandchars=\\\{\}]
\PY{c+c1}{\PYZsh{} ================================================================}
\PY{c+c1}{\PYZsh{} — Procesamiento de imágenes con HE y CLAHE (comparación)}
\PY{c+c1}{\PYZsh{} \PYZhy{}\PYZhy{}\PYZhy{}\PYZhy{}\PYZhy{}\PYZhy{}\PYZhy{}\PYZhy{}\PYZhy{}\PYZhy{}\PYZhy{}\PYZhy{}\PYZhy{}\PYZhy{}\PYZhy{}\PYZhy{}\PYZhy{}\PYZhy{}\PYZhy{}\PYZhy{}\PYZhy{}\PYZhy{}\PYZhy{}\PYZhy{}\PYZhy{}\PYZhy{}\PYZhy{}\PYZhy{}\PYZhy{}\PYZhy{}\PYZhy{}\PYZhy{}\PYZhy{}\PYZhy{}\PYZhy{}\PYZhy{}\PYZhy{}\PYZhy{}\PYZhy{}\PYZhy{}\PYZhy{}\PYZhy{}\PYZhy{}\PYZhy{}\PYZhy{}\PYZhy{}\PYZhy{}\PYZhy{}\PYZhy{}\PYZhy{}\PYZhy{}\PYZhy{}\PYZhy{}\PYZhy{}\PYZhy{}\PYZhy{}\PYZhy{}\PYZhy{}\PYZhy{}\PYZhy{}\PYZhy{}\PYZhy{}\PYZhy{}}
\PY{c+c1}{\PYZsh{} En esta sección se implementa la comparación entre:}
\PY{c+c1}{\PYZsh{}   1. La imagen original en escala de grises.}
\PY{c+c1}{\PYZsh{}   2. Histogram Equalization (HE) global.}
\PY{c+c1}{\PYZsh{}   3. CLAHE con configuración por defecto.}
\PY{c+c1}{\PYZsh{}   4. CLAHE con configuración alternativa (clip limit mayor y bloques más pequeños).}
\PY{c+c1}{\PYZsh{}}
\PY{c+c1}{\PYZsh{} El objetivo es observar cómo varía la distribución del histograma}
\PY{c+c1}{\PYZsh{} y los indicadores cuantitativos (desviación estándar y entropía),}
\PY{c+c1}{\PYZsh{} mostrando en una misma figura la imagen resultante y su histograma.}
\PY{c+c1}{\PYZsh{} ================================================================}

\PY{c+c1}{\PYZsh{} \PYZhy{}\PYZhy{}\PYZhy{}\PYZhy{}\PYZhy{}\PYZhy{}\PYZhy{}\PYZhy{}\PYZhy{}\PYZhy{}\PYZhy{}\PYZhy{}\PYZhy{}\PYZhy{}\PYZhy{}\PYZhy{}\PYZhy{}\PYZhy{}\PYZhy{}\PYZhy{}\PYZhy{}\PYZhy{}\PYZhy{}\PYZhy{}\PYZhy{}\PYZhy{}\PYZhy{}\PYZhy{}\PYZhy{}\PYZhy{}}
\PY{c+c1}{\PYZsh{} Funciones auxiliares}
\PY{c+c1}{\PYZsh{} \PYZhy{}\PYZhy{}\PYZhy{}\PYZhy{}\PYZhy{}\PYZhy{}\PYZhy{}\PYZhy{}\PYZhy{}\PYZhy{}\PYZhy{}\PYZhy{}\PYZhy{}\PYZhy{}\PYZhy{}\PYZhy{}\PYZhy{}\PYZhy{}\PYZhy{}\PYZhy{}\PYZhy{}\PYZhy{}\PYZhy{}\PYZhy{}\PYZhy{}\PYZhy{}\PYZhy{}\PYZhy{}\PYZhy{}\PYZhy{}}

\PY{k}{def}\PY{+w}{ }\PY{n+nf}{to\PYZus{}uint8}\PY{p}{(}\PY{n}{img}\PY{p}{)}\PY{p}{:}
\PY{+w}{    }\PY{l+s+sd}{\PYZdq{}\PYZdq{}\PYZdq{}}
\PY{l+s+sd}{    Convierte la imagen a formato uint8 [0,255].}
\PY{l+s+sd}{    Si la imagen se encuentra en flotantes [0,1], realiza la conversión.}
\PY{l+s+sd}{    Esto asegura compatibilidad con OpenCV en las operaciones posteriores.}
\PY{l+s+sd}{    \PYZdq{}\PYZdq{}\PYZdq{}}
    \PY{k}{return} \PY{p}{(}\PY{n}{img} \PY{o}{*} \PY{l+m+mi}{255}\PY{p}{)}\PY{o}{.}\PY{n}{astype}\PY{p}{(}\PY{n}{np}\PY{o}{.}\PY{n}{uint8}\PY{p}{)} \PY{k}{if} \PY{n}{img}\PY{o}{.}\PY{n}{dtype}\PY{o}{.}\PY{n}{kind} \PY{o}{==} \PY{l+s+s2}{\PYZdq{}}\PY{l+s+s2}{f}\PY{l+s+s2}{\PYZdq{}} \PY{k}{else} \PY{n}{img}\PY{o}{.}\PY{n}{astype}\PY{p}{(}\PY{n}{np}\PY{o}{.}\PY{n}{uint8}\PY{p}{)}

\PY{k}{def}\PY{+w}{ }\PY{n+nf}{image\PYZus{}entropy}\PY{p}{(}\PY{n}{img\PYZus{}uint8}\PY{p}{)}\PY{p}{:}
\PY{+w}{    }\PY{l+s+sd}{\PYZdq{}\PYZdq{}\PYZdq{}}
\PY{l+s+sd}{    Calcula la entropía del histograma de la imagen.}
\PY{l+s+sd}{    La entropía mide el grado de dispersión de los niveles de gris,}
\PY{l+s+sd}{    reflejando la cantidad de información contenida en la imagen.}
\PY{l+s+sd}{    Valores más altos implican mayor diversidad tonal.}
\PY{l+s+sd}{    \PYZdq{}\PYZdq{}\PYZdq{}}
    \PY{n}{hist} \PY{o}{=} \PY{n}{cv2}\PY{o}{.}\PY{n}{calcHist}\PY{p}{(}\PY{p}{[}\PY{n}{img\PYZus{}uint8}\PY{p}{]}\PY{p}{,} \PY{p}{[}\PY{l+m+mi}{0}\PY{p}{]}\PY{p}{,} \PY{k+kc}{None}\PY{p}{,} \PY{p}{[}\PY{l+m+mi}{256}\PY{p}{]}\PY{p}{,} \PY{p}{[}\PY{l+m+mi}{0}\PY{p}{,} \PY{l+m+mi}{256}\PY{p}{]}\PY{p}{)}\PY{o}{.}\PY{n}{ravel}\PY{p}{(}\PY{p}{)}
    \PY{n}{p} \PY{o}{=} \PY{n}{hist} \PY{o}{/} \PY{p}{(}\PY{n}{hist}\PY{o}{.}\PY{n}{sum}\PY{p}{(}\PY{p}{)} \PY{o}{+} \PY{l+m+mf}{1e\PYZhy{}12}\PY{p}{)}   \PY{c+c1}{\PYZsh{} normalización a distribución de probabilidad}
    \PY{n}{p} \PY{o}{=} \PY{n}{p}\PY{p}{[}\PY{n}{p} \PY{o}{\PYZgt{}} \PY{l+m+mi}{0}\PY{p}{]}                      \PY{c+c1}{\PYZsh{} se eliminan ceros para evitar log indefinidos}
    \PY{k}{return} \PY{n+nb}{float}\PY{p}{(}\PY{o}{\PYZhy{}}\PY{p}{(}\PY{n}{p} \PY{o}{*} \PY{n}{np}\PY{o}{.}\PY{n}{log}\PY{p}{(}\PY{n}{p}\PY{p}{)}\PY{p}{)}\PY{o}{.}\PY{n}{sum}\PY{p}{(}\PY{p}{)}\PY{p}{)}

\PY{k}{def}\PY{+w}{ }\PY{n+nf}{show\PYZus{}hist}\PY{p}{(}\PY{n}{ax}\PY{p}{,} \PY{n}{img}\PY{p}{,} \PY{n}{title}\PY{p}{)}\PY{p}{:}
\PY{+w}{    }\PY{l+s+sd}{\PYZdq{}\PYZdq{}\PYZdq{}}
\PY{l+s+sd}{    Grafica el histograma de una imagen en escala de grises.}
\PY{l+s+sd}{    Se utiliza para visualizar la redistribución de intensidades}
\PY{l+s+sd}{    después de aplicar HE o CLAHE.}
\PY{l+s+sd}{    \PYZdq{}\PYZdq{}\PYZdq{}}
    \PY{n}{ax}\PY{o}{.}\PY{n}{hist}\PY{p}{(}\PY{n}{img}\PY{o}{.}\PY{n}{ravel}\PY{p}{(}\PY{p}{)}\PY{p}{,} \PY{n}{bins}\PY{o}{=}\PY{l+m+mi}{256}\PY{p}{,} \PY{n+nb}{range}\PY{o}{=}\PY{p}{(}\PY{l+m+mi}{0}\PY{p}{,} \PY{l+m+mi}{255}\PY{p}{)}\PY{p}{)}
    \PY{n}{ax}\PY{o}{.}\PY{n}{set\PYZus{}title}\PY{p}{(}\PY{n}{title}\PY{p}{)}
    \PY{n}{ax}\PY{o}{.}\PY{n}{set\PYZus{}xlabel}\PY{p}{(}\PY{l+s+s2}{\PYZdq{}}\PY{l+s+s2}{Gray level}\PY{l+s+s2}{\PYZdq{}}\PY{p}{)}
    \PY{n}{ax}\PY{o}{.}\PY{n}{set\PYZus{}ylabel}\PY{p}{(}\PY{l+s+s2}{\PYZdq{}}\PY{l+s+s2}{Frequency}\PY{l+s+s2}{\PYZdq{}}\PY{p}{)}
    \PY{n}{ax}\PY{o}{.}\PY{n}{grid}\PY{p}{(}\PY{n}{alpha}\PY{o}{=}\PY{l+m+mf}{0.3}\PY{p}{)}

\PY{c+c1}{\PYZsh{} \PYZhy{}\PYZhy{}\PYZhy{}\PYZhy{}\PYZhy{}\PYZhy{}\PYZhy{}\PYZhy{}\PYZhy{}\PYZhy{}\PYZhy{}\PYZhy{}\PYZhy{}\PYZhy{}\PYZhy{}\PYZhy{}\PYZhy{}\PYZhy{}\PYZhy{}\PYZhy{}\PYZhy{}\PYZhy{}\PYZhy{}\PYZhy{}\PYZhy{}\PYZhy{}\PYZhy{}\PYZhy{}\PYZhy{}\PYZhy{}}
\PY{c+c1}{\PYZsh{} Conversión a escala de grises}
\PY{c+c1}{\PYZsh{} \PYZhy{}\PYZhy{}\PYZhy{}\PYZhy{}\PYZhy{}\PYZhy{}\PYZhy{}\PYZhy{}\PYZhy{}\PYZhy{}\PYZhy{}\PYZhy{}\PYZhy{}\PYZhy{}\PYZhy{}\PYZhy{}\PYZhy{}\PYZhy{}\PYZhy{}\PYZhy{}\PYZhy{}\PYZhy{}\PYZhy{}\PYZhy{}\PYZhy{}\PYZhy{}\PYZhy{}\PYZhy{}\PYZhy{}\PYZhy{}}
\PY{c+c1}{\PYZsh{} Las técnicas de HE y CLAHE operan sobre imágenes en un solo canal.}
\PY{c+c1}{\PYZsh{} Se convierte cada imagen a escala de grises para aplicar las transformaciones.}
\PY{c+c1}{\PYZsh{} \PYZhy{}\PYZhy{}\PYZhy{}\PYZhy{}\PYZhy{}\PYZhy{}\PYZhy{}\PYZhy{}\PYZhy{}\PYZhy{}\PYZhy{}\PYZhy{}\PYZhy{}\PYZhy{}\PYZhy{}\PYZhy{}\PYZhy{}\PYZhy{}\PYZhy{}\PYZhy{}\PYZhy{}\PYZhy{}\PYZhy{}\PYZhy{}\PYZhy{}\PYZhy{}\PYZhy{}\PYZhy{}\PYZhy{}\PYZhy{}}
\PY{k}{assert} \PY{n}{img7} \PY{o+ow}{is} \PY{o+ow}{not} \PY{k+kc}{None} \PY{o+ow}{and} \PY{n}{img8} \PY{o+ow}{is} \PY{o+ow}{not} \PY{k+kc}{None} \PY{o+ow}{and} \PY{n}{img9} \PY{o+ow}{is} \PY{o+ow}{not} \PY{k+kc}{None}\PY{p}{,} \PY{l+s+s2}{\PYZdq{}}\PY{l+s+s2}{Las imágenes deben cargarse primero.}\PY{l+s+s2}{\PYZdq{}}

\PY{n}{imgs\PYZus{}rgb}  \PY{o}{=} \PY{p}{[}\PY{n}{img7}\PY{p}{,} \PY{n}{img8}\PY{p}{,} \PY{n}{img9}\PY{p}{]}
\PY{n}{titles}    \PY{o}{=} \PY{p}{[}\PY{l+s+s2}{\PYZdq{}}\PY{l+s+s2}{Image 7 — Low Light}\PY{l+s+s2}{\PYZdq{}}\PY{p}{,} \PY{l+s+s2}{\PYZdq{}}\PY{l+s+s2}{Image 8 — Flowers}\PY{l+s+s2}{\PYZdq{}}\PY{p}{,} \PY{l+s+s2}{\PYZdq{}}\PY{l+s+s2}{Image 9 — Wall}\PY{l+s+s2}{\PYZdq{}}\PY{p}{]}
\PY{n}{imgs\PYZus{}gray} \PY{o}{=} \PY{p}{[}\PY{n}{cv2}\PY{o}{.}\PY{n}{cvtColor}\PY{p}{(}\PY{n}{to\PYZus{}uint8}\PY{p}{(}\PY{n}{im}\PY{p}{)}\PY{p}{,} \PY{n}{cv2}\PY{o}{.}\PY{n}{COLOR\PYZus{}RGB2GRAY}\PY{p}{)} \PY{k}{for} \PY{n}{im} \PY{o+ow}{in} \PY{n}{imgs\PYZus{}rgb}\PY{p}{]}

\PY{c+c1}{\PYZsh{} \PYZhy{}\PYZhy{}\PYZhy{}\PYZhy{}\PYZhy{}\PYZhy{}\PYZhy{}\PYZhy{}\PYZhy{}\PYZhy{}\PYZhy{}\PYZhy{}\PYZhy{}\PYZhy{}\PYZhy{}\PYZhy{}\PYZhy{}\PYZhy{}\PYZhy{}\PYZhy{}\PYZhy{}\PYZhy{}\PYZhy{}\PYZhy{}\PYZhy{}\PYZhy{}\PYZhy{}\PYZhy{}\PYZhy{}\PYZhy{}}
\PY{c+c1}{\PYZsh{} Configuración de CLAHE}
\PY{c+c1}{\PYZsh{} \PYZhy{}\PYZhy{}\PYZhy{}\PYZhy{}\PYZhy{}\PYZhy{}\PYZhy{}\PYZhy{}\PYZhy{}\PYZhy{}\PYZhy{}\PYZhy{}\PYZhy{}\PYZhy{}\PYZhy{}\PYZhy{}\PYZhy{}\PYZhy{}\PYZhy{}\PYZhy{}\PYZhy{}\PYZhy{}\PYZhy{}\PYZhy{}\PYZhy{}\PYZhy{}\PYZhy{}\PYZhy{}\PYZhy{}\PYZhy{}}
\PY{c+c1}{\PYZsh{} \PYZhy{} Configuración por defecto: clipLimit=2.0, grid=(8x8).}
\PY{c+c1}{\PYZsh{} \PYZhy{} Variante: clipLimit=4.0, grid=(4x4).}
\PY{c+c1}{\PYZsh{} Estas configuraciones permiten comparar cómo la limitación de contraste}
\PY{c+c1}{\PYZsh{} y el tamaño de las teselas afectan la mejora de la imagen.}
\PY{c+c1}{\PYZsh{} \PYZhy{}\PYZhy{}\PYZhy{}\PYZhy{}\PYZhy{}\PYZhy{}\PYZhy{}\PYZhy{}\PYZhy{}\PYZhy{}\PYZhy{}\PYZhy{}\PYZhy{}\PYZhy{}\PYZhy{}\PYZhy{}\PYZhy{}\PYZhy{}\PYZhy{}\PYZhy{}\PYZhy{}\PYZhy{}\PYZhy{}\PYZhy{}\PYZhy{}\PYZhy{}\PYZhy{}\PYZhy{}\PYZhy{}\PYZhy{}}
\PY{n}{clahe\PYZus{}cfg\PYZus{}default} \PY{o}{=} \PY{p}{\PYZob{}}\PY{l+s+s2}{\PYZdq{}}\PY{l+s+s2}{clip}\PY{l+s+s2}{\PYZdq{}}\PY{p}{:} \PY{l+m+mf}{2.0}\PY{p}{,} \PY{l+s+s2}{\PYZdq{}}\PY{l+s+s2}{grid}\PY{l+s+s2}{\PYZdq{}}\PY{p}{:} \PY{p}{(}\PY{l+m+mi}{8}\PY{p}{,} \PY{l+m+mi}{8}\PY{p}{)}\PY{p}{\PYZcb{}}
\PY{n}{clahe\PYZus{}cfg\PYZus{}variant} \PY{o}{=} \PY{p}{\PYZob{}}\PY{l+s+s2}{\PYZdq{}}\PY{l+s+s2}{clip}\PY{l+s+s2}{\PYZdq{}}\PY{p}{:} \PY{l+m+mf}{4.0}\PY{p}{,} \PY{l+s+s2}{\PYZdq{}}\PY{l+s+s2}{grid}\PY{l+s+s2}{\PYZdq{}}\PY{p}{:} \PY{p}{(}\PY{l+m+mi}{4}\PY{p}{,} \PY{l+m+mi}{4}\PY{p}{)}\PY{p}{\PYZcb{}}

\PY{c+c1}{\PYZsh{} \PYZhy{}\PYZhy{}\PYZhy{}\PYZhy{}\PYZhy{}\PYZhy{}\PYZhy{}\PYZhy{}\PYZhy{}\PYZhy{}\PYZhy{}\PYZhy{}\PYZhy{}\PYZhy{}\PYZhy{}\PYZhy{}\PYZhy{}\PYZhy{}\PYZhy{}\PYZhy{}\PYZhy{}\PYZhy{}\PYZhy{}\PYZhy{}\PYZhy{}\PYZhy{}\PYZhy{}\PYZhy{}\PYZhy{}\PYZhy{}}
\PY{c+c1}{\PYZsh{} Procesamiento principal}
\PY{c+c1}{\PYZsh{} \PYZhy{}\PYZhy{}\PYZhy{}\PYZhy{}\PYZhy{}\PYZhy{}\PYZhy{}\PYZhy{}\PYZhy{}\PYZhy{}\PYZhy{}\PYZhy{}\PYZhy{}\PYZhy{}\PYZhy{}\PYZhy{}\PYZhy{}\PYZhy{}\PYZhy{}\PYZhy{}\PYZhy{}\PYZhy{}\PYZhy{}\PYZhy{}\PYZhy{}\PYZhy{}\PYZhy{}\PYZhy{}\PYZhy{}\PYZhy{}}
\PY{c+c1}{\PYZsh{} Para cada imagen:}
\PY{c+c1}{\PYZsh{}   1. Se calculan métricas del original (desviación estándar y entropía).}
\PY{c+c1}{\PYZsh{}   2. Se aplica HE (global).}
\PY{c+c1}{\PYZsh{}   3. Se aplica CLAHE en dos configuraciones.}
\PY{c+c1}{\PYZsh{}   4. Se muestran las imágenes y sus histogramas.}
\PY{c+c1}{\PYZsh{}   5. Se almacenan las métricas en una tabla comparativa.}
\PY{c+c1}{\PYZsh{} \PYZhy{}\PYZhy{}\PYZhy{}\PYZhy{}\PYZhy{}\PYZhy{}\PYZhy{}\PYZhy{}\PYZhy{}\PYZhy{}\PYZhy{}\PYZhy{}\PYZhy{}\PYZhy{}\PYZhy{}\PYZhy{}\PYZhy{}\PYZhy{}\PYZhy{}\PYZhy{}\PYZhy{}\PYZhy{}\PYZhy{}\PYZhy{}\PYZhy{}\PYZhy{}\PYZhy{}\PYZhy{}\PYZhy{}\PYZhy{}}
\PY{n}{summary\PYZus{}rows} \PY{o}{=} \PY{p}{[}\PY{p}{]}

\PY{k}{for} \PY{n}{gimg}\PY{p}{,} \PY{n}{title} \PY{o+ow}{in} \PY{n+nb}{zip}\PY{p}{(}\PY{n}{imgs\PYZus{}gray}\PY{p}{,} \PY{n}{titles}\PY{p}{)}\PY{p}{:}

    \PY{c+c1}{\PYZsh{} \PYZhy{}\PYZhy{}\PYZhy{} Imagen original \PYZhy{}\PYZhy{}\PYZhy{}}
    \PY{n}{orig\PYZus{}std} \PY{o}{=} \PY{n+nb}{float}\PY{p}{(}\PY{n}{np}\PY{o}{.}\PY{n}{std}\PY{p}{(}\PY{n}{gimg}\PY{p}{)}\PY{p}{)}
    \PY{n}{orig\PYZus{}ent} \PY{o}{=} \PY{n}{image\PYZus{}entropy}\PY{p}{(}\PY{n}{gimg}\PY{p}{)}

    \PY{c+c1}{\PYZsh{} \PYZhy{}\PYZhy{}\PYZhy{} Histogram Equalization (HE) global \PYZhy{}\PYZhy{}\PYZhy{}}
    \PY{n}{he\PYZus{}img}  \PY{o}{=} \PY{n}{cv2}\PY{o}{.}\PY{n}{equalizeHist}\PY{p}{(}\PY{n}{gimg}\PY{p}{)}
    \PY{n}{he\PYZus{}std}  \PY{o}{=} \PY{n+nb}{float}\PY{p}{(}\PY{n}{np}\PY{o}{.}\PY{n}{std}\PY{p}{(}\PY{n}{he\PYZus{}img}\PY{p}{)}\PY{p}{)}
    \PY{n}{he\PYZus{}ent}  \PY{o}{=} \PY{n}{image\PYZus{}entropy}\PY{p}{(}\PY{n}{he\PYZus{}img}\PY{p}{)}

    \PY{c+c1}{\PYZsh{} \PYZhy{}\PYZhy{}\PYZhy{} CLAHE (configuración por defecto) \PYZhy{}\PYZhy{}\PYZhy{}}
    \PY{n}{clahe\PYZus{}d} \PY{o}{=} \PY{n}{cv2}\PY{o}{.}\PY{n}{createCLAHE}\PY{p}{(}\PY{n}{clipLimit}\PY{o}{=}\PY{n}{clahe\PYZus{}cfg\PYZus{}default}\PY{p}{[}\PY{l+s+s2}{\PYZdq{}}\PY{l+s+s2}{clip}\PY{l+s+s2}{\PYZdq{}}\PY{p}{]}\PY{p}{,}
                              \PY{n}{tileGridSize}\PY{o}{=}\PY{n}{clahe\PYZus{}cfg\PYZus{}default}\PY{p}{[}\PY{l+s+s2}{\PYZdq{}}\PY{l+s+s2}{grid}\PY{l+s+s2}{\PYZdq{}}\PY{p}{]}\PY{p}{)}\PY{o}{.}\PY{n}{apply}\PY{p}{(}\PY{n}{gimg}\PY{p}{)}
    \PY{n}{d\PYZus{}std}   \PY{o}{=} \PY{n+nb}{float}\PY{p}{(}\PY{n}{np}\PY{o}{.}\PY{n}{std}\PY{p}{(}\PY{n}{clahe\PYZus{}d}\PY{p}{)}\PY{p}{)}
    \PY{n}{d\PYZus{}ent}   \PY{o}{=} \PY{n}{image\PYZus{}entropy}\PY{p}{(}\PY{n}{clahe\PYZus{}d}\PY{p}{)}

    \PY{c+c1}{\PYZsh{} \PYZhy{}\PYZhy{}\PYZhy{} CLAHE (configuración variante) \PYZhy{}\PYZhy{}\PYZhy{}}
    \PY{n}{clahe\PYZus{}v} \PY{o}{=} \PY{n}{cv2}\PY{o}{.}\PY{n}{createCLAHE}\PY{p}{(}\PY{n}{clipLimit}\PY{o}{=}\PY{n}{clahe\PYZus{}cfg\PYZus{}variant}\PY{p}{[}\PY{l+s+s2}{\PYZdq{}}\PY{l+s+s2}{clip}\PY{l+s+s2}{\PYZdq{}}\PY{p}{]}\PY{p}{,}
                              \PY{n}{tileGridSize}\PY{o}{=}\PY{n}{clahe\PYZus{}cfg\PYZus{}variant}\PY{p}{[}\PY{l+s+s2}{\PYZdq{}}\PY{l+s+s2}{grid}\PY{l+s+s2}{\PYZdq{}}\PY{p}{]}\PY{p}{)}\PY{o}{.}\PY{n}{apply}\PY{p}{(}\PY{n}{gimg}\PY{p}{)}
    \PY{n}{v\PYZus{}std}   \PY{o}{=} \PY{n+nb}{float}\PY{p}{(}\PY{n}{np}\PY{o}{.}\PY{n}{std}\PY{p}{(}\PY{n}{clahe\PYZus{}v}\PY{p}{)}\PY{p}{)}
    \PY{n}{v\PYZus{}ent}   \PY{o}{=} \PY{n}{image\PYZus{}entropy}\PY{p}{(}\PY{n}{clahe\PYZus{}v}\PY{p}{)}

    \PY{c+c1}{\PYZsh{} \PYZhy{}\PYZhy{}\PYZhy{} Visualización comparativa \PYZhy{}\PYZhy{}\PYZhy{}}
    \PY{c+c1}{\PYZsh{} Se generan figuras con 2 filas × 4 columnas:}
    \PY{c+c1}{\PYZsh{}   Fila 1: imágenes (Original, HE, CLAHE default, CLAHE variant).}
    \PY{c+c1}{\PYZsh{}   Fila 2: histogramas correspondientes.}
    \PY{n}{fig}\PY{p}{,} \PY{n}{axes} \PY{o}{=} \PY{n}{plt}\PY{o}{.}\PY{n}{subplots}\PY{p}{(}\PY{l+m+mi}{2}\PY{p}{,} \PY{l+m+mi}{4}\PY{p}{,} \PY{n}{figsize}\PY{o}{=}\PY{p}{(}\PY{l+m+mi}{18}\PY{p}{,} \PY{l+m+mi}{7}\PY{p}{)}\PY{p}{)}
    \PY{n}{fig}\PY{o}{.}\PY{n}{suptitle}\PY{p}{(}\PY{l+s+sa}{f}\PY{l+s+s2}{\PYZdq{}}\PY{l+s+s2}{HE vs CLAHE — }\PY{l+s+si}{\PYZob{}}\PY{n}{title}\PY{l+s+si}{\PYZcb{}}\PY{l+s+s2}{\PYZdq{}}\PY{p}{,} \PY{n}{fontsize}\PY{o}{=}\PY{l+m+mi}{14}\PY{p}{,} \PY{n}{fontweight}\PY{o}{=}\PY{l+s+s2}{\PYZdq{}}\PY{l+s+s2}{bold}\PY{l+s+s2}{\PYZdq{}}\PY{p}{)}

    \PY{c+c1}{\PYZsh{} Fila 1: imágenes}
    \PY{n}{axes}\PY{p}{[}\PY{l+m+mi}{0}\PY{p}{,} \PY{l+m+mi}{0}\PY{p}{]}\PY{o}{.}\PY{n}{imshow}\PY{p}{(}\PY{n}{gimg}\PY{p}{,} \PY{n}{cmap}\PY{o}{=}\PY{l+s+s2}{\PYZdq{}}\PY{l+s+s2}{gray}\PY{l+s+s2}{\PYZdq{}}\PY{p}{)}\PY{p}{;}    \PY{n}{axes}\PY{p}{[}\PY{l+m+mi}{0}\PY{p}{,} \PY{l+m+mi}{0}\PY{p}{]}\PY{o}{.}\PY{n}{set\PYZus{}title}\PY{p}{(}\PY{l+s+s2}{\PYZdq{}}\PY{l+s+s2}{Original}\PY{l+s+s2}{\PYZdq{}}\PY{p}{)}\PY{p}{;} \PY{n}{axes}\PY{p}{[}\PY{l+m+mi}{0}\PY{p}{,} \PY{l+m+mi}{0}\PY{p}{]}\PY{o}{.}\PY{n}{axis}\PY{p}{(}\PY{l+s+s2}{\PYZdq{}}\PY{l+s+s2}{off}\PY{l+s+s2}{\PYZdq{}}\PY{p}{)}
    \PY{n}{axes}\PY{p}{[}\PY{l+m+mi}{0}\PY{p}{,} \PY{l+m+mi}{1}\PY{p}{]}\PY{o}{.}\PY{n}{imshow}\PY{p}{(}\PY{n}{he\PYZus{}img}\PY{p}{,} \PY{n}{cmap}\PY{o}{=}\PY{l+s+s2}{\PYZdq{}}\PY{l+s+s2}{gray}\PY{l+s+s2}{\PYZdq{}}\PY{p}{)}\PY{p}{;}  \PY{n}{axes}\PY{p}{[}\PY{l+m+mi}{0}\PY{p}{,} \PY{l+m+mi}{1}\PY{p}{]}\PY{o}{.}\PY{n}{set\PYZus{}title}\PY{p}{(}\PY{l+s+s2}{\PYZdq{}}\PY{l+s+s2}{HE (Global)}\PY{l+s+s2}{\PYZdq{}}\PY{p}{)}\PY{p}{;} \PY{n}{axes}\PY{p}{[}\PY{l+m+mi}{0}\PY{p}{,} \PY{l+m+mi}{1}\PY{p}{]}\PY{o}{.}\PY{n}{axis}\PY{p}{(}\PY{l+s+s2}{\PYZdq{}}\PY{l+s+s2}{off}\PY{l+s+s2}{\PYZdq{}}\PY{p}{)}
    \PY{n}{axes}\PY{p}{[}\PY{l+m+mi}{0}\PY{p}{,} \PY{l+m+mi}{2}\PY{p}{]}\PY{o}{.}\PY{n}{imshow}\PY{p}{(}\PY{n}{clahe\PYZus{}d}\PY{p}{,} \PY{n}{cmap}\PY{o}{=}\PY{l+s+s2}{\PYZdq{}}\PY{l+s+s2}{gray}\PY{l+s+s2}{\PYZdq{}}\PY{p}{)}\PY{p}{;} \PY{n}{axes}\PY{p}{[}\PY{l+m+mi}{0}\PY{p}{,} \PY{l+m+mi}{2}\PY{p}{]}\PY{o}{.}\PY{n}{set\PYZus{}title}\PY{p}{(}\PY{l+s+sa}{f}\PY{l+s+s2}{\PYZdq{}}\PY{l+s+s2}{CLAHE default}\PY{l+s+se}{\PYZbs{}n}\PY{l+s+s2}{clip=}\PY{l+s+si}{\PYZob{}}\PY{n}{clahe\PYZus{}cfg\PYZus{}default}\PY{p}{[}\PY{l+s+s1}{\PYZsq{}}\PY{l+s+s1}{clip}\PY{l+s+s1}{\PYZsq{}}\PY{p}{]}\PY{l+s+si}{\PYZcb{}}\PY{l+s+s2}{, grid=}\PY{l+s+si}{\PYZob{}}\PY{n}{clahe\PYZus{}cfg\PYZus{}default}\PY{p}{[}\PY{l+s+s1}{\PYZsq{}}\PY{l+s+s1}{grid}\PY{l+s+s1}{\PYZsq{}}\PY{p}{]}\PY{l+s+si}{\PYZcb{}}\PY{l+s+s2}{\PYZdq{}}\PY{p}{)}\PY{p}{;} \PY{n}{axes}\PY{p}{[}\PY{l+m+mi}{0}\PY{p}{,} \PY{l+m+mi}{2}\PY{p}{]}\PY{o}{.}\PY{n}{axis}\PY{p}{(}\PY{l+s+s2}{\PYZdq{}}\PY{l+s+s2}{off}\PY{l+s+s2}{\PYZdq{}}\PY{p}{)}
    \PY{n}{axes}\PY{p}{[}\PY{l+m+mi}{0}\PY{p}{,} \PY{l+m+mi}{3}\PY{p}{]}\PY{o}{.}\PY{n}{imshow}\PY{p}{(}\PY{n}{clahe\PYZus{}v}\PY{p}{,} \PY{n}{cmap}\PY{o}{=}\PY{l+s+s2}{\PYZdq{}}\PY{l+s+s2}{gray}\PY{l+s+s2}{\PYZdq{}}\PY{p}{)}\PY{p}{;} \PY{n}{axes}\PY{p}{[}\PY{l+m+mi}{0}\PY{p}{,} \PY{l+m+mi}{3}\PY{p}{]}\PY{o}{.}\PY{n}{set\PYZus{}title}\PY{p}{(}\PY{l+s+sa}{f}\PY{l+s+s2}{\PYZdq{}}\PY{l+s+s2}{CLAHE variant}\PY{l+s+se}{\PYZbs{}n}\PY{l+s+s2}{clip=}\PY{l+s+si}{\PYZob{}}\PY{n}{clahe\PYZus{}cfg\PYZus{}variant}\PY{p}{[}\PY{l+s+s1}{\PYZsq{}}\PY{l+s+s1}{clip}\PY{l+s+s1}{\PYZsq{}}\PY{p}{]}\PY{l+s+si}{\PYZcb{}}\PY{l+s+s2}{, grid=}\PY{l+s+si}{\PYZob{}}\PY{n}{clahe\PYZus{}cfg\PYZus{}variant}\PY{p}{[}\PY{l+s+s1}{\PYZsq{}}\PY{l+s+s1}{grid}\PY{l+s+s1}{\PYZsq{}}\PY{p}{]}\PY{l+s+si}{\PYZcb{}}\PY{l+s+s2}{\PYZdq{}}\PY{p}{)}\PY{p}{;} \PY{n}{axes}\PY{p}{[}\PY{l+m+mi}{0}\PY{p}{,} \PY{l+m+mi}{3}\PY{p}{]}\PY{o}{.}\PY{n}{axis}\PY{p}{(}\PY{l+s+s2}{\PYZdq{}}\PY{l+s+s2}{off}\PY{l+s+s2}{\PYZdq{}}\PY{p}{)}

    \PY{c+c1}{\PYZsh{} Fila 2: histogramas}
    \PY{n}{show\PYZus{}hist}\PY{p}{(}\PY{n}{axes}\PY{p}{[}\PY{l+m+mi}{1}\PY{p}{,} \PY{l+m+mi}{0}\PY{p}{]}\PY{p}{,} \PY{n}{gimg}\PY{p}{,}   \PY{l+s+sa}{f}\PY{l+s+s2}{\PYZdq{}}\PY{l+s+s2}{Original}\PY{l+s+se}{\PYZbs{}n}\PY{l+s+s2}{std=}\PY{l+s+si}{\PYZob{}}\PY{n}{orig\PYZus{}std}\PY{l+s+si}{:}\PY{l+s+s2}{.2f}\PY{l+s+si}{\PYZcb{}}\PY{l+s+s2}{  ent=}\PY{l+s+si}{\PYZob{}}\PY{n}{orig\PYZus{}ent}\PY{l+s+si}{:}\PY{l+s+s2}{.3f}\PY{l+s+si}{\PYZcb{}}\PY{l+s+s2}{\PYZdq{}}\PY{p}{)}
    \PY{n}{show\PYZus{}hist}\PY{p}{(}\PY{n}{axes}\PY{p}{[}\PY{l+m+mi}{1}\PY{p}{,} \PY{l+m+mi}{1}\PY{p}{]}\PY{p}{,} \PY{n}{he\PYZus{}img}\PY{p}{,} \PY{l+s+sa}{f}\PY{l+s+s2}{\PYZdq{}}\PY{l+s+s2}{HE}\PY{l+s+se}{\PYZbs{}n}\PY{l+s+s2}{std=}\PY{l+s+si}{\PYZob{}}\PY{n}{he\PYZus{}std}\PY{l+s+si}{:}\PY{l+s+s2}{.2f}\PY{l+s+si}{\PYZcb{}}\PY{l+s+s2}{  ent=}\PY{l+s+si}{\PYZob{}}\PY{n}{he\PYZus{}ent}\PY{l+s+si}{:}\PY{l+s+s2}{.3f}\PY{l+s+si}{\PYZcb{}}\PY{l+s+s2}{\PYZdq{}}\PY{p}{)}
    \PY{n}{show\PYZus{}hist}\PY{p}{(}\PY{n}{axes}\PY{p}{[}\PY{l+m+mi}{1}\PY{p}{,} \PY{l+m+mi}{2}\PY{p}{]}\PY{p}{,} \PY{n}{clahe\PYZus{}d}\PY{p}{,}\PY{l+s+sa}{f}\PY{l+s+s2}{\PYZdq{}}\PY{l+s+s2}{CLAHE default}\PY{l+s+se}{\PYZbs{}n}\PY{l+s+s2}{std=}\PY{l+s+si}{\PYZob{}}\PY{n}{d\PYZus{}std}\PY{l+s+si}{:}\PY{l+s+s2}{.2f}\PY{l+s+si}{\PYZcb{}}\PY{l+s+s2}{  ent=}\PY{l+s+si}{\PYZob{}}\PY{n}{d\PYZus{}ent}\PY{l+s+si}{:}\PY{l+s+s2}{.3f}\PY{l+s+si}{\PYZcb{}}\PY{l+s+s2}{\PYZdq{}}\PY{p}{)}
    \PY{n}{show\PYZus{}hist}\PY{p}{(}\PY{n}{axes}\PY{p}{[}\PY{l+m+mi}{1}\PY{p}{,} \PY{l+m+mi}{3}\PY{p}{]}\PY{p}{,} \PY{n}{clahe\PYZus{}v}\PY{p}{,}\PY{l+s+sa}{f}\PY{l+s+s2}{\PYZdq{}}\PY{l+s+s2}{CLAHE variant}\PY{l+s+se}{\PYZbs{}n}\PY{l+s+s2}{std=}\PY{l+s+si}{\PYZob{}}\PY{n}{v\PYZus{}std}\PY{l+s+si}{:}\PY{l+s+s2}{.2f}\PY{l+s+si}{\PYZcb{}}\PY{l+s+s2}{  ent=}\PY{l+s+si}{\PYZob{}}\PY{n}{v\PYZus{}ent}\PY{l+s+si}{:}\PY{l+s+s2}{.3f}\PY{l+s+si}{\PYZcb{}}\PY{l+s+s2}{\PYZdq{}}\PY{p}{)}

    \PY{n}{plt}\PY{o}{.}\PY{n}{tight\PYZus{}layout}\PY{p}{(}\PY{p}{)}
    \PY{n}{plt}\PY{o}{.}\PY{n}{show}\PY{p}{(}\PY{p}{)}

    \PY{c+c1}{\PYZsh{} \PYZhy{}\PYZhy{}\PYZhy{} Registro de resultados para la tabla comparativa \PYZhy{}\PYZhy{}\PYZhy{}}
    \PY{n}{summary\PYZus{}rows} \PY{o}{+}\PY{o}{=} \PY{p}{[}
        \PY{p}{(}\PY{n}{title}\PY{p}{,} \PY{l+s+s2}{\PYZdq{}}\PY{l+s+s2}{Original}\PY{l+s+s2}{\PYZdq{}}\PY{p}{,}       \PY{n}{orig\PYZus{}std}\PY{p}{,} \PY{n}{orig\PYZus{}ent}\PY{p}{,} \PY{l+s+s2}{\PYZdq{}}\PY{l+s+s2}{—}\PY{l+s+s2}{\PYZdq{}}\PY{p}{,} \PY{l+s+s2}{\PYZdq{}}\PY{l+s+s2}{—}\PY{l+s+s2}{\PYZdq{}}\PY{p}{)}\PY{p}{,}
        \PY{p}{(}\PY{n}{title}\PY{p}{,} \PY{l+s+s2}{\PYZdq{}}\PY{l+s+s2}{HE (global)}\PY{l+s+s2}{\PYZdq{}}\PY{p}{,}    \PY{n}{he\PYZus{}std}\PY{p}{,}   \PY{n}{he\PYZus{}ent}\PY{p}{,}   \PY{l+s+s2}{\PYZdq{}}\PY{l+s+s2}{—}\PY{l+s+s2}{\PYZdq{}}\PY{p}{,} \PY{l+s+s2}{\PYZdq{}}\PY{l+s+s2}{—}\PY{l+s+s2}{\PYZdq{}}\PY{p}{)}\PY{p}{,}
        \PY{p}{(}\PY{n}{title}\PY{p}{,} \PY{l+s+s2}{\PYZdq{}}\PY{l+s+s2}{CLAHE default}\PY{l+s+s2}{\PYZdq{}}\PY{p}{,}  \PY{n}{d\PYZus{}std}\PY{p}{,}    \PY{n}{d\PYZus{}ent}\PY{p}{,}    \PY{l+s+sa}{f}\PY{l+s+s2}{\PYZdq{}}\PY{l+s+s2}{clip=}\PY{l+s+si}{\PYZob{}}\PY{n}{clahe\PYZus{}cfg\PYZus{}default}\PY{p}{[}\PY{l+s+s1}{\PYZsq{}}\PY{l+s+s1}{clip}\PY{l+s+s1}{\PYZsq{}}\PY{p}{]}\PY{l+s+si}{\PYZcb{}}\PY{l+s+s2}{\PYZdq{}}\PY{p}{,}  \PY{l+s+sa}{f}\PY{l+s+s2}{\PYZdq{}}\PY{l+s+s2}{grid=}\PY{l+s+si}{\PYZob{}}\PY{n}{clahe\PYZus{}cfg\PYZus{}default}\PY{p}{[}\PY{l+s+s1}{\PYZsq{}}\PY{l+s+s1}{grid}\PY{l+s+s1}{\PYZsq{}}\PY{p}{]}\PY{l+s+si}{\PYZcb{}}\PY{l+s+s2}{\PYZdq{}}\PY{p}{)}\PY{p}{,}
        \PY{p}{(}\PY{n}{title}\PY{p}{,} \PY{l+s+s2}{\PYZdq{}}\PY{l+s+s2}{CLAHE variant}\PY{l+s+s2}{\PYZdq{}}\PY{p}{,}  \PY{n}{v\PYZus{}std}\PY{p}{,}    \PY{n}{v\PYZus{}ent}\PY{p}{,}    \PY{l+s+sa}{f}\PY{l+s+s2}{\PYZdq{}}\PY{l+s+s2}{clip=}\PY{l+s+si}{\PYZob{}}\PY{n}{clahe\PYZus{}cfg\PYZus{}variant}\PY{p}{[}\PY{l+s+s1}{\PYZsq{}}\PY{l+s+s1}{clip}\PY{l+s+s1}{\PYZsq{}}\PY{p}{]}\PY{l+s+si}{\PYZcb{}}\PY{l+s+s2}{\PYZdq{}}\PY{p}{,}  \PY{l+s+sa}{f}\PY{l+s+s2}{\PYZdq{}}\PY{l+s+s2}{grid=}\PY{l+s+si}{\PYZob{}}\PY{n}{clahe\PYZus{}cfg\PYZus{}variant}\PY{p}{[}\PY{l+s+s1}{\PYZsq{}}\PY{l+s+s1}{grid}\PY{l+s+s1}{\PYZsq{}}\PY{p}{]}\PY{l+s+si}{\PYZcb{}}\PY{l+s+s2}{\PYZdq{}}\PY{p}{)}\PY{p}{,}
    \PY{p}{]}

\PY{c+c1}{\PYZsh{} \PYZhy{}\PYZhy{}\PYZhy{}\PYZhy{}\PYZhy{}\PYZhy{}\PYZhy{}\PYZhy{}\PYZhy{}\PYZhy{}\PYZhy{}\PYZhy{}\PYZhy{}\PYZhy{}\PYZhy{}\PYZhy{}\PYZhy{}\PYZhy{}\PYZhy{}\PYZhy{}\PYZhy{}\PYZhy{}\PYZhy{}\PYZhy{}\PYZhy{}\PYZhy{}\PYZhy{}\PYZhy{}\PYZhy{}\PYZhy{}}
\PY{c+c1}{\PYZsh{} Tabla resumen de resultados}
\PY{c+c1}{\PYZsh{} \PYZhy{}\PYZhy{}\PYZhy{}\PYZhy{}\PYZhy{}\PYZhy{}\PYZhy{}\PYZhy{}\PYZhy{}\PYZhy{}\PYZhy{}\PYZhy{}\PYZhy{}\PYZhy{}\PYZhy{}\PYZhy{}\PYZhy{}\PYZhy{}\PYZhy{}\PYZhy{}\PYZhy{}\PYZhy{}\PYZhy{}\PYZhy{}\PYZhy{}\PYZhy{}\PYZhy{}\PYZhy{}\PYZhy{}\PYZhy{}}
\PY{c+c1}{\PYZsh{} Se presenta un cuadro con:}
\PY{c+c1}{\PYZsh{}   \PYZhy{} Nombre de la imagen.}
\PY{c+c1}{\PYZsh{}   \PYZhy{} Método aplicado.}
\PY{c+c1}{\PYZsh{}   \PYZhy{} Desviación estándar (std).}
\PY{c+c1}{\PYZsh{}   \PYZhy{} Entropía (ent).}
\PY{c+c1}{\PYZsh{}   \PYZhy{} Parámetros de CLAHE (si aplican).}
\PY{c+c1}{\PYZsh{} \PYZhy{}\PYZhy{}\PYZhy{}\PYZhy{}\PYZhy{}\PYZhy{}\PYZhy{}\PYZhy{}\PYZhy{}\PYZhy{}\PYZhy{}\PYZhy{}\PYZhy{}\PYZhy{}\PYZhy{}\PYZhy{}\PYZhy{}\PYZhy{}\PYZhy{}\PYZhy{}\PYZhy{}\PYZhy{}\PYZhy{}\PYZhy{}\PYZhy{}\PYZhy{}\PYZhy{}\PYZhy{}\PYZhy{}\PYZhy{}}
\PY{n+nb}{print}\PY{p}{(}\PY{l+s+s2}{\PYZdq{}}\PY{l+s+se}{\PYZbs{}n}\PY{l+s+s2}{=== Comparison Summary (std / entropy) ===}\PY{l+s+s2}{\PYZdq{}}\PY{p}{)}
\PY{n+nb}{print}\PY{p}{(}\PY{l+s+sa}{f}\PY{l+s+s2}{\PYZdq{}}\PY{l+s+si}{\PYZob{}}\PY{l+s+s1}{\PYZsq{}}\PY{l+s+s1}{Image}\PY{l+s+s1}{\PYZsq{}}\PY{l+s+si}{:}\PY{l+s+s2}{\PYZlt{}22}\PY{l+s+si}{\PYZcb{}}\PY{l+s+s2}{ | }\PY{l+s+si}{\PYZob{}}\PY{l+s+s1}{\PYZsq{}}\PY{l+s+s1}{Method}\PY{l+s+s1}{\PYZsq{}}\PY{l+s+si}{:}\PY{l+s+s2}{\PYZlt{}15}\PY{l+s+si}{\PYZcb{}}\PY{l+s+s2}{ | }\PY{l+s+si}{\PYZob{}}\PY{l+s+s1}{\PYZsq{}}\PY{l+s+s1}{std}\PY{l+s+s1}{\PYZsq{}}\PY{l+s+si}{:}\PY{l+s+s2}{\PYZgt{}8}\PY{l+s+si}{\PYZcb{}}\PY{l+s+s2}{ | }\PY{l+s+si}{\PYZob{}}\PY{l+s+s1}{\PYZsq{}}\PY{l+s+s1}{entropy}\PY{l+s+s1}{\PYZsq{}}\PY{l+s+si}{:}\PY{l+s+s2}{\PYZgt{}8}\PY{l+s+si}{\PYZcb{}}\PY{l+s+s2}{ | }\PY{l+s+si}{\PYZob{}}\PY{l+s+s1}{\PYZsq{}}\PY{l+s+s1}{clip}\PY{l+s+s1}{\PYZsq{}}\PY{l+s+si}{:}\PY{l+s+s2}{\PYZlt{}10}\PY{l+s+si}{\PYZcb{}}\PY{l+s+s2}{ | }\PY{l+s+si}{\PYZob{}}\PY{l+s+s1}{\PYZsq{}}\PY{l+s+s1}{grid}\PY{l+s+s1}{\PYZsq{}}\PY{l+s+si}{:}\PY{l+s+s2}{\PYZlt{}10}\PY{l+s+si}{\PYZcb{}}\PY{l+s+s2}{\PYZdq{}}\PY{p}{)}
\PY{n+nb}{print}\PY{p}{(}\PY{l+s+s2}{\PYZdq{}}\PY{l+s+s2}{\PYZhy{}}\PY{l+s+s2}{\PYZdq{}}\PY{o}{*}\PY{l+m+mi}{80}\PY{p}{)}
\PY{k}{for} \PY{n}{row} \PY{o+ow}{in} \PY{n}{summary\PYZus{}rows}\PY{p}{:}
    \PY{n}{img\PYZus{}name}\PY{p}{,} \PY{n}{method}\PY{p}{,} \PY{n}{s}\PY{p}{,} \PY{n}{e}\PY{p}{,} \PY{n}{clip}\PY{p}{,} \PY{n}{grid} \PY{o}{=} \PY{n}{row}
    \PY{n+nb}{print}\PY{p}{(}\PY{l+s+sa}{f}\PY{l+s+s2}{\PYZdq{}}\PY{l+s+si}{\PYZob{}}\PY{n}{img\PYZus{}name}\PY{l+s+si}{:}\PY{l+s+s2}{\PYZlt{}22}\PY{l+s+si}{\PYZcb{}}\PY{l+s+s2}{ | }\PY{l+s+si}{\PYZob{}}\PY{n}{method}\PY{l+s+si}{:}\PY{l+s+s2}{\PYZlt{}15}\PY{l+s+si}{\PYZcb{}}\PY{l+s+s2}{ | }\PY{l+s+si}{\PYZob{}}\PY{n}{s}\PY{l+s+si}{:}\PY{l+s+s2}{8.2f}\PY{l+s+si}{\PYZcb{}}\PY{l+s+s2}{ | }\PY{l+s+si}{\PYZob{}}\PY{n}{e}\PY{l+s+si}{:}\PY{l+s+s2}{8.3f}\PY{l+s+si}{\PYZcb{}}\PY{l+s+s2}{ | }\PY{l+s+si}{\PYZob{}}\PY{n}{clip}\PY{l+s+si}{:}\PY{l+s+s2}{\PYZlt{}10}\PY{l+s+si}{\PYZcb{}}\PY{l+s+s2}{ | }\PY{l+s+si}{\PYZob{}}\PY{n}{grid}\PY{l+s+si}{:}\PY{l+s+s2}{\PYZlt{}10}\PY{l+s+si}{\PYZcb{}}\PY{l+s+s2}{\PYZdq{}}\PY{p}{)}
\end{Verbatim}
\end{tcolorbox}

    \begin{center}
    \adjustimage{max size={0.9\linewidth}{0.9\paperheight}}{output_21_0.png}
    \end{center}
    { \hspace*{\fill} \\}
    
    \begin{center}
    \adjustimage{max size={0.9\linewidth}{0.9\paperheight}}{output_21_1.png}
    \end{center}
    { \hspace*{\fill} \\}
    
    \begin{center}
    \adjustimage{max size={0.9\linewidth}{0.9\paperheight}}{output_21_2.png}
    \end{center}
    { \hspace*{\fill} \\}
    
    \begin{Verbatim}[commandchars=\\\{\}]

=== Comparison Summary (std / entropy) ===
Image                  | Method          |      std |  entropy | clip       |
grid
--------------------------------------------------------------------------------
Image 7 — Low Light    | Original        |    23.15 |    2.373 | —          | —
Image 7 — Low Light    | HE (global)     |    78.18 |    2.244 | —          | —
Image 7 — Low Light    | CLAHE default   |    33.46 |    2.922 | clip=2.0   |
grid=(8, 8)
Image 7 — Low Light    | CLAHE variant   |    34.97 |    2.919 | clip=4.0   |
grid=(4, 4)
Image 8 — Flowers      | Original        |    79.92 |    5.440 | —          | —
Image 8 — Flowers      | HE (global)     |    73.65 |    5.277 | —          | —
Image 8 — Flowers      | CLAHE default   |    74.36 |    5.530 | clip=2.0   |
grid=(8, 8)
Image 8 — Flowers      | CLAHE variant   |    72.76 |    5.532 | clip=4.0   |
grid=(4, 4)
Image 9 — Wall         | Original        |     8.56 |    3.519 | —          | —
Image 9 — Wall         | HE (global)     |    74.24 |    3.487 | —          | —
Image 9 — Wall         | CLAHE default   |    12.50 |    3.919 | clip=2.0   |
grid=(8, 8)
Image 9 — Wall         | CLAHE variant   |    19.15 |    4.341 | clip=4.0   |
grid=(4, 4)
    \end{Verbatim}

    \hypertarget{results}{%
\subsection{Results}\label{results}}

The comparative analysis between global Histogram Equalization (HE) and
Contrast Limited Adaptive Histogram Equalization (CLAHE) is presented
through three case studies: a scene under low-light conditions (Image
7), a set of flowers with naturally high contrast and strong color
presence (Image 8), and a homogeneous wall with low contrast (Image 9).
The evaluation metrics employed were the standard deviation (σ),
associated with the global contrast of the image, and entropy (H),
related to the amount of information and the distribution of gray
levels.

Across the three test cases, the results show that global Histogram
Equalization (HE) tends to excessively stretch the histogram, yielding
very high standard deviation values (e.g., σ = 78.18 in the low-light
image and σ = 74.24 in the homogeneous wall), but often at the expense
of entropy reduction, indicating loss of tonal diversity and the
introduction of visual artifacts. In contrast, CLAHE consistently
produced a more balanced enhancement: in low-light and homogeneous
images it increased both contrast and entropy (e.g., from H = 2.373 to H
≈ 2.92 in the low-light case, and from H = 3.519 to H = 4.341 in the
wall), revealing hidden details without saturation; while in the
high-contrast flower image, CLAHE preserved and slightly improved
entropy (H = 5.532) without degrading the natural tonal distribution,
unlike HE which reduced it (H = 5.277). Overall, the numerical evidence
confirms that CLAHE achieves controlled local contrast enhancement and
better information retention compared to the global and often unstable
behavior of HE.

The experimental comparison between global Histogram Equalization (HE)
and Contrast Limited Adaptive Histogram Equalization (CLAHE)
demonstrates that while HE is effective in globally redistributing pixel
intensities, it often leads to over-amplification of contrast and a
reduction in entropy, which results in loss of tonal diversity and the
introduction of artifacts, particularly in homogeneous or low-light
images (Wang \& Yang, 2024). On the other hand, CLAHE consistently
provides a more controlled and adaptive enhancement, improving both
contrast and entropy in challenging scenarios while preserving natural
tonal distributions in already high-contrast images (Mohammed \& Isa,
2025; Soni \& Waoo, 2025). These results confirm that CLAHE is a more
robust and versatile approach for practical applications, particularly
in domains such as medical imaging, low-light photography, and remote
sensing, where the preservation of fine details and controlled noise
amplification are critical.

    \hypertarget{bibliography}{%
\subsection{Bibliography}\label{bibliography}}

\begin{itemize}
\tightlist
\item
  \textbf{Zuiderveld, K. (1994). Contrast-limited adaptive histogram
  equalization. In \emph{Graphics Gems IV} (pp.~474--485). Academic
  Press.}
\item
  \textbf{OpenCV Documentation.}
\item
  \textbf{Digital Image Processing, 4th Edition, by Rafael C. Gonzalez
  and Richard E. Woods.}
\item
  \textbf{Majid Mohammed, I., \& Ashidi Mat Isa, N. (2025). Contrast
  Limited Adaptive Local Histogram Equalization Method for Poor Contrast
  Image Enhancement. IEEE Access, 13, 62600--62632.
  https://doi.org/10.1109/ACCESS.2025.3558506}
\item
  \textbf{Soni, L. N., \& Waoo, A. A. (2025). Face Detection Under
  Low-Light and Low-Resolution Conditions Using Contrast-Limited
  Adaptive Histogram Equalization and a Modified Convolutional Neural
  Network. Journal of Neonatal Surgery, 14(32s), 2620--2631.
  https://doi.org/10.63682/jns.v14i32s.7793}
\item
  \textbf{Wang, W., \& Yang, Y. (2024). A histogram equalization model
  for color image contrast enhancement. Signal, Image and Video
  Processing, 18(2), 1725--1732.
  https://doi.org/10.1007/s11760-023-02881-9}
\item
  \textbf{Ochoa Ruiz, G. (2025). Ecualización por histogramas. Computer
  Science Dept., Advanced AI Research Group, Tecnológico de Monterrey.
  {[}Class presentation{]}.}
\end{itemize}


    % Add a bibliography block to the postdoc
    
    
    
\end{document}
