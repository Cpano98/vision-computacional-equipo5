\documentclass[12pt,letterpaper]{article}

% --- PAQUETES TIPOGRÁFICOS Y BÁSICOS ---
\usepackage[utf8]{inputenc}
\usepackage[spanish]{babel}
\usepackage{microtype}            % Mejora la justificación
\usepackage{newtxtext,newtxmath}  % Tipografía elegante compatible con pdfLaTeX
\usepackage{graphicx}
\usepackage{amsmath}
\usepackage{geometry}
\usepackage{hyperref}
\usepackage{caption}
\usepackage{enumitem}
\usepackage{fancyhdr}
\usepackage{titlesec}
\usepackage{xcolor}
\usepackage{float}     % para usar [H]
\usepackage{placeins}  % para \FloatBarrier

% --- GEOMETRÍA DE PÁGINA ---
\geometry{
  letterpaper,
  margin=1in,
  headheight=15pt % Corrige warning de fancyhdr
}

% --- PÁRRAFOS Y LÍNEA ---
\setlength{\parindent}{1.5em}   % Sangría inicial en párrafos
\setlength{\parskip}{0.4em}     % Pequeño espacio entre párrafos
\linespread{1.05}               % Ligeramente más aire entre líneas
\raggedbottom                   % Evita estiramientos verticales feos

% --- LISTAS COMPACTAS (sin perder legibilidad) ---
\setlist[itemize]{topsep=4pt,itemsep=2pt,parsep=2pt}
\setlist[enumerate]{topsep=4pt,itemsep=2pt,parsep=2pt}

% --- CAPTIONS ---
\captionsetup[figure]{labelfont=bf,font=small}
\captionsetup[table]{labelfont=bf,font=small}

% --- HYPERREF ---
\hypersetup{
  colorlinks=true,
  linkcolor=blue!50!black,
  urlcolor=teal!60!black,
  citecolor=purple!60!black,
  pdftitle={<<Titulo del documento>>},
  pdfpagemode=UseOutlines
}

% --- ENCABEZADOS Y PIES ---
\pagestyle{fancy}
\fancyhf{}
\fancyhead[L]{$<<Titulo del documento>>$}
\fancyhead[R]{\thepage}
\renewcommand{\headrulewidth}{0.4pt}
\renewcommand{\footrulewidth}{0pt}
\setlength{\headheight}{15pt} % Corrige el warning de fancyhdr

% --- ESTILO DE SECCIONES ---
\titleformat{\section}{\large\bfseries}{\thesection}{0.6em}{}
\titleformat{\subsection}{\normalsize\bfseries}{\thesubsection}{0.6em}{}
\titleformat{\subsubsection}{\normalsize\itshape}{\thesubsubsection}{0.6em}{}
\titlespacing*{\section}{0pt}{0.8em}{0.4em}
\titlespacing*{\subsection}{0pt}{0.6em}{0.3em}
\titlespacing*{\subsubsection}{0pt}{0.5em}{0.25em}

% --- RUTA DE IMÁGENES (opcional) ---
% \graphicspath{{figuras/}}

% --- MODO DRAFT PARA COMPILACIÓN RÁPIDA ---
% Si tienes problemas de timeout, descomenta la siguiente línea:
% \usepackage[draft]{graphicx}  % Esto evita cargar las imágenes completas

% ==============================
% INICIO DEL DOCUMENTO
% ==============================
\begin{document}

% ========== CARÁTULA ==========
\begin{titlepage}
  \centering
  \vspace*{0.5cm}
  \includegraphics[width=0.5\textwidth]{logo.jpg}\par
  \vspace{1.2cm}

  {\Large\scshape Instituto Tecnológico y de Estudios Superiores de Monterrey\par}
  \vspace{2.2cm}

  {\huge\bfseries Actividad <<NumeroDeActividad>>\par
   \vspace{0.2cm}
   \emph{<<Nombre de la Actividad>>}\par}
  \vspace{2.2cm}

  {\Large\bfseries Equipo 5\par}
  {\normalsize\bfseries A01796323 Benjamín Cisneros Barraza\par}
  {\normalsize\bfseries A01066264 Carlos Pano Hernandez\par}
  {\normalsize\bfseries A01795590 Edgar Omar Cruz Mendoza\par}
  {\normalsize\bfseries A01275322 Jonatan Israel Meza Mendoza\par}
  
  \vspace{1.2cm}

  \begin{flushleft}
    \normalsize \textbf{Profesor Titular:} Dr. Gilberto Ochoa Ruiz\\
    \normalsize \textbf{Profesor Asistente:} MIP Ma. del Refugio Meléndez Alfaro\\
    \normalsize \textbf{Profesor Asistente:} Iván Reyes Amezcua
  \end{flushleft}

  \vfill

  \begin{flushright}
    \normalsize Visión computacional para imágenes y video\\
    \normalsize <<dia>> de <<mes>> de 2025
  \end{flushright}
\end{titlepage}

\setcounter{page}{1} % Reinicia numeración tras la carátula

% ========== RESUMEN ==========
\section*{Resumen}

<<Resumen del trabajo>>
% ========== OBJETIVOS ==========
\section*{Objetivos del <<TipoDocumento>>}

\begin{enumerate}
  \item <<Objetivo1>>
  \item <<Objetivo2>>
  \item <<Objetivo3>>
\end{enumerate}

\newpage

% ========== TABLA DE CONTENIDOS ==========
\tableofcontents
\newpage

% ========== TABLA DE FIGURAS ==========
\listoffigures
\newpage

% ========== INTRODUCCIÓN ==========
\section{Introducción}

<<Intro>>

\newpage

\section{<<Tema1>>}

<<Texto>>

\subsection{<<Subtema>>}
<<Texto>>

\subsection{<<Subtema>>}

<<Texto>>

\begin{itemize}
  \item \textbf{<<Elemento1>>}: <<DescripcionElemento1>>
  \item \textbf{<<Elemento2>>}: <<DescripcionElemento2>>
  \item \textbf{<<Elemento3>>}: <<DescripcionElemento3>>
\end{itemize}

<<Texto>>


\newpage

\section{<<Tema2>>}

<<Texto>>

\subsection{<<Subtema>>}

<<Texto>>

<<Texto>>

Referencia a imagen \ref{fig:imagen_1}

\begin{figure}[H]
  \centering
  \includegraphics[width=0.5\linewidth]{imagen_1.jpg}
  \caption{<<DescripcionImagen>>}
  \label{fig:imagen_1}
\end{figure}

<<Tabla>>

\begin{table}[H]
\centering
\begin{tabular}{|p{3.5cm}|p{3.5cm}|p{3.5cm}|p{3.5cm}|}
\hline
\textbf{<<Columna1>>} & \textbf{<<Columna2>>} & \textbf{<<Columna3>>} & \textbf{<<Columna4>>} \\
\hline
<<Dato1Col1>> & <<Dato1Col2>> & <<Dato1Col3>> & <<Dato1Col3>> \\
\hline
<<Dato2Col1>> & <<Dato2Col2>> & <<Dato2Col3>> & <<Dato2Col3>> \\
\hline
<<Dato3Col1>> & <<Dato3Col2>> & <<Dato3Col3>> & <<Dato3Col3>> \\
\hline
\end{tabular}
\caption{<<DescripcionTabla>>}
\end{table}

\bigskip

\noindent
<<Texto>>
\newpage

\section{Conclusiones}

<<Colocar las conclusiones>>

\newpage

% ==============================
% REFERENCIAS - SOLO UNA VEZ
% ==============================
\begin{thebibliography}{99}

\bibitem{aguilar2020}
Aguilar-Alvarado, J. V., \& Campoverde-Molina, M. A. (2020). Clasificación de frutas basadas en redes neuronales convolucionales. \textit{Polo del Conocimiento, 5(1)}, 3–22. \url{https://dialnet.unirioja.es/descarga/articulo/7436055.pdf}

\bibitem{calonder2010}
Calonder, M., Lepetit, V., Strecha, C., \& Fua, P. (2010). BRIEF: Binary robust independent elementary features. \textit{European Conference on Computer Vision}, 778--792.

\bibitem{chen2018}
Chen, L. C., Zhu, Y., Papandreou, G., Schroff, F., \& Adam, H. (2018). Encoder-decoder with atrous separable convolution for semantic image segmentation. \textit{Proceedings of the European Conference on Computer Vision (ECCV)}, 801--818.

\bibitem{gonzalez2018}
Gonzalez, R. C., \& Woods, R. E. (2018). \textit{Digital image processing} (4th ed.). Pearson.

\bibitem{gruponortenita2025}
Grupo La Norteñita. (s.f.). \textit{Empaque}. \url{https://www.grupolanortenita.com/empaque}

\bibitem{he2017}
He, K., Gkioxari, G., Dollár, P., \& Girshick, R. (2017). Mask R-CNN. \textit{Proceedings of the IEEE International Conference on Computer Vision}, 2961--2969.

\bibitem{lowe2004}
Lowe, D. G. (2004). Distinctive image features from scale-invariant keypoints. \textit{International Journal of Computer Vision}, 60(2), 91--110.

\bibitem{mafroda}
MAF RODA. (s.f.). \textit{Su tecnología Globalscan 7 gana en fiabilidad, sencillez y precisión}. Revista Mercados. \url{https://www.3-22.eu/maf-roda-peaches-globalscan-7/}

\bibitem{montoya2014}
Montoya Holguín, C., Cortés Osorio, J. A., \& Chaves Osorio, J. A. (2014). Sistema automático de reconocimiento de frutas basado en visión por computador. Ingeniare. \textit{Revista Chilena de Ingeniería, 22(4)}, 504–516. \url{https://doi.org/10.4067/S0718-33052014000400006}

\bibitem{ochoa2023a}
Ochoa, G. (2023a). \textit{Bienvenida al curso} [Video]. Instituto Tecnológico y de Estudios Superiores de Monterrey. Visión computacional para imágenes y video.

\bibitem{ochoa2023b}
Ochoa, G. (2023b). \textit{Módulo 1. Introducción y objetivos} [Video]. Instituto Tecnológico y de Estudios Superiores de Monterrey. Visión computacional para imágenes y video.

\bibitem{ochoa2023c}
Ochoa, G. (2023c). \textit{PPT: Visión computacional para imágenes y video - Módulo 1, Tema 1.1 Introducción} [Presentación de PowerPoint]. Instituto Tecnológico y de Estudios Superiores de Monterrey.

\bibitem{rosten2006}
Rosten, E., \& Drummond, T. (2006). Machine learning for high-speed corner detection. \textit{European Conference on Computer Vision}, 430--443.

\bibitem{rublee2011}
Rublee, E., Rabaud, V., Konolige, K., \& Bradski, G. (2011). ORB: An efficient alternative to SIFT or SURF. \textit{2011 International Conference on Computer Vision}, 2564--2571.

\bibitem{szeliski2022}
Szeliski, R. (2022). \textit{Computer vision: Algorithms and applications} (2nd ed.). Springer.


\end{thebibliography}

% ==============================
% NOTAS PARA RESOLVER PROBLEMAS DE COMPILACIÓN
% ==============================
% Si experimentas timeout en la compilación:
% 
% 1. IMÁGENES: Optimiza las imágenes antes de subirlas
%    - Reduce la resolución a 150-300 DPI máximo
%    - Usa formatos comprimidos: JPG para fotos (calidad 80-90%)
%    - Redimensiona a tamaño real de uso (no más de 1500px de ancho)
%    - Herramientas: ImageMagick, TinyPNG, o Squoosh.app
%
% 2. COMPILACIÓN RÁPIDA: Para pruebas, usa modo draft
%    - Cambia \usepackage{graphicx} por \usepackage[draft]{graphicx}
%    - Esto muestra solo los marcos de las imágenes
%
% 3. SI PERSISTE EL PROBLEMA:
%    - Compila localmente con TeXLive o MiKTeX
%    - Usa un plan de pago en Overleaf para más tiempo de compilación
%    - Divide el documento en archivos separados con \input{}

\end{document}